\documentclass{scrartcl}
\usepackage{german}
\usepackage[utf8]{inputenc}
\usepackage[german]{babel}
\usepackage{amssymb} % what does it do?
\usepackage{graphicx} % I can't do that yet
\usepackage{fancyhdr} % what does it do?
\usepackage{lastpage} % what does it do?
\setlength{\parskip}{\medskipamount} % thats reasonable
\setlength{\parindent}{0pt} % whatever that does


%%%%%%%%%%%%%%%%%%%%%%%%
% Kopf- und Fusszeilen %
%%%%%%%%%%%%%%%%%%%%%%%%
\pagestyle{fancy}
\lhead{
    \begin{tabular}{ll}
        Felix Karg & 4342014\\
    \end{tabular}
}
\chead{Info II - AlgoDat}
\rhead{
    \begin{tabular}{rr}
        \today{} \\
        Seite \thepage{} von \pageref{LastPage}
    \end{tabular}
}
\lfoot{}
\cfoot{}
\rfoot{}

%%%%%%%%%%%%%%%%%%%%%%%%
% Anfang des Dokuments %
%%%%%%%%%%%%%%%%%%%%%%%%
\begin{document}

\section*{Antworten zu Übungsblatt Nr. 11}


\section*{Aufgabe 1}

Es wurde ja bereits in der VL gezeigt, dass es eine Folge von Operation gibt,
die minimal (optimal) ist, und |ED(x, y)| Operationen enthält. Nun ist zu
zeigen dass es mindestens einer dieser optimalen Folgen gibt, die monoton ist,
oder sich jede in diese Optimale überführen lässt.

Beweis 1: \\
Wenn man das Diagramm nach VL 11b (Iterativer Algorithmus) mit allen (kürzesten)
varianten betrachtet, bzw die Wege der Kürzesten Editierdistanz von unten Rechts
in umgekehrter Reihnfolge der Operationen, so sind all diese (auf der ebene der
Kürzesten Editierdistanz) optimal (da Kürzeste Editierdistanz), sowie monoton
(Eindeutig aufsteigende Indizes).


Beweis 2: \\
Man nehme eine optimale Folge F von Operationen mit länge k = ED(x, y).
Man ordne nun den Operationen als Index den Index der Operation in der
Zeichenkette. Haben nun keine zwei Operationen den angeblich selben index
sind wir Fertig, falls doch: Die Operationen sind entweder ein oder Mehrere
Delete's sowie ein möglicherweise darauffolgendes Replace. Nun werden alle
Darauffolgenden Operationen ans Ende (Indexmäßig) 'verschoben', sowie das
mögliche Replace als erstes mit dem Eigentlichen Index (falls vorhanden)
(sowie auch der Index des Replacement, falls dies nötig war muss auch der
Index des Delete an dieser stelle auf den ursprünglichen des Replace gesetzt
werden), und alle darauffolgenden Deletes an eine jeweils um eins erhöhte stelle.

(Ich hoffe es ist erkenntlich was gemeint ist ... Beweis 1 Funktioniert notfalls auch ...)


\section*{Aufgabe 2}
Wir wissen dass die Editierdistanz zwischen dem Absolutwert der Distanz der beiden sowie
max(|x|, |y|) liegt. Wir setzen also anfangs $\delta = abs(|x| - |y|)$, und gehen berechnen
die jeweils nächste Nebendiagonale sollte dies nicht ausreichen, bis maximal max(|x|, |y|).


\end{document}
