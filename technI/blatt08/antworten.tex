\documentclass{scrartcl}
\usepackage{german}
\usepackage[T1]{fontenc}
\usepackage[latin1]{inputenc}
\usepackage[german]{babel}

% zusätzliche mathematische Symbole, AMS=American Mathematical Society
\usepackage{amssymb}

% fürs Einbinden von Graphiken
\usepackage{graphicx}

% für Namen etc. in Kopf- oder Fußzeile
\usepackage{fancyhdr}

% erlaubt benutzerdefinierte Kopfzeilen
\pagestyle{fancy}

% Definition der Kopfzeile
\lhead{
	\begin{tabular}{ll}
		Fisnik Zeqiri & 4306430 \\
		Felix  Karg   & 4342014
	\end{tabular}
}
\chead{}
\rhead{\today{}}
\lfoot{}
\cfoot{Seite \thepage}
\rfoot{}

\begin{document}
	\section*{Antworten zum �bungsblatt Nr. 8}

	\section*{Aufgabe 1}
	\begin{itemize}
	    \item[a)] Tabelle:\

		\begin{tabular}{l|l|l|l|l}
			$ s^{t}$ & $a^{t}$ & $b^{t}$ & $s^{t+1}$ & $y^{t} $ \\\hline
			s00                 & 1                   & 0                   & s01                   & 1                   \\
			s00                 & 1                   & 1                   & s01                   & 1                   \\
			s00                 & 0                   & 0                   & s10                   & 0                   \\
			s00                 & 0                   & 1                   & s10                   & 0                   \\
			s01                 & 1                   & 1                   & s11                   & 1                   \\
			s01                 & 0                   & 1                   & s00                   & 1                   \\
			s01                 & 0                   & 0                   & s00                   & 1                   \\
			s01                 & 1                   & 0                   & s00                   & 1                   \\
			s11                 & 0                   & 0                   & s10                   & 0                   \\
			s11                 & 0                   & 1                   & s10                   & 0                   \\
			s11                 & 1                   & 0                   & s10                   & 0                   \\
			s11                 & 1                   & 1                   & s10                   & 0                   \\
			s10                 & 0                   & 1                   & s01                   & 0                   \\
			s10                 & 1                   & 0                   & s00                   & 1                   \\
			s10                 & 1                   & 1                   & s00                   & 1                   \\
			s10                 & 0                   & 0                   & s11                   & 1
		\end{tabular}

	    \item[b)] Polynome f�r Zustandsvariablen: \\
                $ s_{00} := s_{01}*(a' + b') + s_{10}a$ \\
                $ s_{01} := s_{00}a + s_{10}a'b$ \\
                $ s_{10} := s_{00}a' + s_{11} $ \\
                $ s_{11} := s_{10}a'b' + s_{01}ab $ \\

	\end{itemize}
	\section*{Aufgabe 2}
	\begin{itemize}
		\item[a)] PCLd, IRd, ALUAd, ASMd, DDid
		\item[b)] ACCDd, PCLd, IRd, ALUAd, ASMd
		\item[c)] IAd, ACCDd, ASMd
		\item[d)] IRd, ACCLd, ALUDId
		\item[e)] ALUDId, ACCLd, DRd, ASMd, IAd
		\item[f)] Nur unter Einschr�nkung realisierbar: entweder 2Takte oder ein neuer Treiber n�tig: IN2Rd\\
		Realisierung mit IN2Rd: IN1Ld, IN2Ld, ALUDId
		\item[g)] IN1Ld, IN2Rd, ALUDId, ACCDd, IAd, ASMd\\
		ACC wird hierbei �berschrieben. Alternativ auch ein neuer Treiber ALUDd statt ACCd und ALUDId verwendbar um ACC nicht zu �berschrriben.
		\item[h)] Nicht realisierbar in einer Executephase. Es sind Zwei Phasen Notwendig. Dem Acc m�sste zwischenzeitlich der Wert zugewiesen werden.\\
		Takt 1: IAd, ASMd, DRd, ACCLd, ALUDId\\
		Takt 2: ACCDd, IAd, ASMd\\
		\\
	\end{itemize}
	\section*{Aufgabe 3}
	$ACCck = s_0's_1E$. \\
	$/ACCDdoe = (E*(s_1 + s_0))'$ \\

        Vorgehensweise: Zeitdiagramm und RETI-Schaltkreis angesehen und nachgedacht.

	\section*{Aufgabe 4}
	\begin{verbatim}

	; S(20) = X
	; S(21) = Y

	LOAD 21         ; Y
	JUMP EQ +13     ; IF Y = 0 THEN JUMP.
	SUB 20          ; ACC = Y - X
	JUMP LE +7      ; IF X > Y THEN JUMP ELSE SORT
	LOAD 21
	STORE 19
	LOAD 20
	STORE 21
	LOAD 19
	STORE 20
	; X > Y.
	LOAD 20         ; ACC = X
	SUB 21          ; ACC = X - Y
	STORE 20        ; X = X - Y
	JUMP -13

	LOAD 20         ; Y = 0; X = GGT
	STORE 22        ; ERGEBNIS = X


	\end{verbatim}


\end{document}
