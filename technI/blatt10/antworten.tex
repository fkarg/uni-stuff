\documentclass{scrartcl}
\usepackage{german}
\usepackage[T1]{fontenc}
\usepackage[latin1]{inputenc}
\usepackage[german]{babel}
\usepackage[dvipsnames]{xcolor}

% zusätzliche mathematische Symbole, AMS=American Mathematical Society
\usepackage{amssymb}

% fürs Einbinden von Graphiken
\usepackage{graphicx}

% für Namen etc. in Kopf- oder Fußzeile
\usepackage{fancyhdr}

% erlaubt benutzerdefinierte Kopfzeilen
\pagestyle{fancy}

\newcommand{\col}[1]{\colorbox{BurntOrange}{\makebox(0,7){#1}}}
\newcommand{\colr}[1]{\colorbox{RubineRed}{\makebox(0,7){#1}}}

% Definition der Kopfzeile
\lhead{
\begin{tabular}{ll}
Fisnik Zeqiri & 4306430 \\
Felix  Karg   & 4342014
\end{tabular}
}
\chead{}
\rhead{\today{}}
\lfoot{}
\cfoot{Seite \thepage}
\rfoot{}

\begin{document}
\section*{Antworten zum �bungsblatt Nr. 10}


\section*{Aufgabe 2}

\col{ } = Gepr�fte Bits \\
\colr{ } = Betrachtetes Parity-bit \\

\begin{itemize}
    \item[a)] 
        1\col{0}0\col{1}X\col{0}1\col{0}X\col{0}X\colr{1} -> Ungerade => \colr{1} \\
        1\col{0}\col{0}1X\col{0}\col{1}0X\col{0}\colr{1}1 -> Ungerade => \colr{1} \\
        \col{1}001X\col{0}\col{1}\col{0}\colr{0}011 -> Gerade => \colr{0} \\
        \col{1}\col{0}\col{0}\col{1}\colr{0}0100011 -> Gerade => \colr{0} \\ \\

        1\col{1}1\col{0}X\col{1}1\col{0}X\col{1}X\colr{1} -> Ungerade => \colr{1}\\
        1\col{1}\col{1}0X\col{1}\col{1}0X\col{1}\colr{1}1 -> Ungerade => \colr{1}\\
        \col{1}110X\col{1}\col{1}\col{0}\colr{1}111 -> Ungerade => \colr{1} \\
        \col{1}\col{1}\col{1}\col{0}\colr{1}1101111 -> Ungerade => \colr{1} \\

    \item[b)]

        0\col{1}1\col{1}0\col{0}0\col{0}0\col{1}1\colr{1} => Passt \\
        0\col{1}\col{1}10\col{0}\col{0}00\col{1}\colr{1}1 => Passt \\
        \col{0}1110\col{0}\col{0}\col{0}\colr{0}111 => Passt \\
        \col{0}\col{1}\col{1}\col{1}\colr{0}0000111 => Fehler in Parity bit. \\
        0111\colr{1}0000111 => Richtiger Code

        1\col{0}1\col{1}0\col{1}1\col{0}1\col{1}1\colr{1} => Passt \\
        1\col{0}\col{1}10\col{1}\col{1}01\col{1}\colr{1}1 => Passt nicht \\
        \col{1}0110\col{1}\col{1}\col{0}\colr{1}111 => Passt \\
        \col{1}\col{0}\col{1}\col{1}\colr{0}1101111 => Passt nicht \\
        10\colr{0}101101111 => Fehler in bit 10, Richtiger Code

\end{itemize}

\newpage

\section*{Aufgabe 3}
\begin{itemize}

    \item[a)]
        Wenn ein Element der Matrix anders ist sind zwingend die zwei 
        jeweiligen Parity-bits auch anders, also insgesamt 3 bits, dist(c) = 3.
        \\ \\
        Beh.: Immer mindestens 1-Fehlerkorrigierend und 1-Fehlererkennend. \\
        Bew.: Entweder ist irgendwo ein normales Bit geflippt. Dann bekommen
        wir �ber die Parity-Bits sch�ne Koordinaten. Ist irgendein Paritybit
        geflippt wissen wir auch dass es dieses ist, da es selbst einen Fehler
        in seiner Reihe/Spalte indiziert, die anderen Paritybits aber einen
        Fehler in ihrer Spalte/Reihe verneinen, somit ist das angeschlagene 
        Paritybit falsch �bertragen. \\ \\
        Sollten nun zwei Bits falsch �bertragen werden, k�nnen wir dies manchmal 
        nicht einmal mehr erkennen. \\
        Bsp.: Ein Element der Matrix und eines der Paritybits in derselben 
        Zeile/Spalte wurden gedreht, nun schl�gt f�r uns sichtbar nur ein
        Paritybit an, was nach unserer ersten Interpretation hei�en w�rde,
        dass allein dieses falsch ist. Wir k�nnten in diesem Fall nicht einmal
        sagen, ob es einen oder mehrere �bertragungsfehler gab. H�ufig ist es
        allerdings m�glich auch mehrere Fehler eindeutig zu erkennen und je nach
        Konstellation auch zu Korrigieren.  \\ \\
        Wir sind also bei 1-Fehlererkennend ($3 \geq 1$) und
        1-Fehlerkorrigierend ($3 \geq 2*1 + 1 = 3$).

    \newpage
\colr{ } = Echter Fehler \\
\col{ } = Erkannter Fehler \\

    \item[b)] (allgemein formuliert in a) \\
        Konkretes Beispiel: \\
        \begin{tabular}{llll|l}
            1 & 0 & 1 & 0 & 0 \\
            0 & 1 & 0 & 0 & 1 \\
            0 & 1 & 1 & 0 & 0 \\
            \hline
            1 & 0 & 0 & 0 & \\
        \end{tabular}

        Zwei �bertragungsfehler: \\
        \begin{tabular}{llll|l}
            1 & 0 & 1 & 0 & 0 \\
            0 & \colr{0} & 0 & 0 & \colr{0} \\
            0 & 1 & 1 & 0 & 0 \\
            \hline
            1 & 0 & 0 & 0 & \\
        \end{tabular}

        1. M�glichkeit f�r Fehler \\
        \begin{tabular}{llll|l}
            1 & 0 & 1 & 0 & 0 \\
            0 & 0 & 0 & 0 & 0 \\
            0 & 1 & 1 & 0 & 0 \\
            \hline
            1 & \col{0} & 0 & 0 & \\
        \end{tabular}

        2. M�glichkeit f�r Fehler \\
        \begin{tabular}{llll|l}
            1 & 0 & 1 & 0 & 0 \\
            0 & \col{0} & 0 & 0 & \col{0} \\
            0 & 1 & 1 & 0 & 0 \\
            \hline
            1 & 0 & 0 & 0 & \\
        \end{tabular}


    \item[c)] Der betrachtete Code $q'$ ist nun in der Lage h�chstens 2 Fehler
        immer zu finden und auch zu Korrigieren. Bei mehr h�ngt es wieder von der
        Konstellation ab. \\ \\

        Bsp.: (erweiterung von b) \\

        Folgendes wird nun durch P erkannt: \\
        \begin{tabular}{llll|l}
            1 & 0 & 1 & 0 & 0 \\
            0 & \col{0} & 0 & 0 & \col{0} \\
            0 & 1 & 1 & 0 & 0 \\
            \hline
            1 & 0 & 0 & 0 & 0 \\
        \end{tabular}

        Folgendes allerdings wieder nicht, hier wird nicht einmal ein Fehler erkannt: \\
        \begin{tabular}{llll|l}
            1 & 0 & 1 & 0 & 0 \\
            0 & \colr{0} & 0 & 0 & \colr{0} \\
            0 & 1 & 1 & 0 & 0 \\
            \hline
            1 & \colr{1} & 0 & 0 & 0 \\
        \end{tabular}




\end{itemize}


\end{document}
