\documentclass{scrartcl}
\usepackage{german}
\usepackage[utf8]{inputenc}
\usepackage[german]{babel}
\usepackage{amssymb} % what does it do?
\usepackage{graphicx} % I can't do that yet
\usepackage{fancyhdr} % what does it do?
\usepackage{lastpage} % what does it do?
\setlength{\parskip}{\medskipamount} % thats reasonable
\setlength{\parindent}{0pt} % whatever that does


%%%%%%%%%%%%%%%%%%%%%%%%
% Kopf- und Fusszeilen %
%%%%%%%%%%%%%%%%%%%%%%%%
\pagestyle{fancy}
\lhead{
    \begin{tabular}{ll}
        Felix Karg & 4342014\\
    \end{tabular}
}
\chead{Systeme II}
\rhead{
    \begin{tabular}{rr}
        \today{} \\
        Seite \thepage{} von \pageref{LastPage}
    \end{tabular}
}
\lfoot{}
\cfoot{}
\rfoot{}

%%%%%%%%%%%%%%%%%%%%%%%%
% Anfang des Dokuments %
%%%%%%%%%%%%%%%%%%%%%%%%
\begin{document}

\section*{Antworten zu Übungsblatt Nr. 3}

\section{Aufgabe}

\subsection{}
\subsection{Involvierte schichten beim Nachrichtenversenden von Rechner (1,2) sowie (1,3)}
Bei der Kommunikation zwischen Rechner 1 und 2 ist nur der Physical Layer wirklich von Bedeutung,
da beide über den Hub verbunden sind. Zwischen 1 und 3 ist schon mehr nötig, von A nach C wäre nur
der Physical Layer notwendig, für die Reibungslose Kommunikation mit dem Switch ist allerdings auch
der Data Link Layer nötig. sobald wir den Router erreichen benötigen wir außerdem dan Network Layer,
um anständig weitergeleitet zu werden. Danach springen wir wieder über einen Switch und einen Hub,
für die wir die benötigten Schichten bereits verwenden. Und so kommen wir bei Rechner 3 an!


\section{Aufgabe: Analoge Signalübertragung}
\subsection{Symbolrate und Datenrate}
Die Symbolrate beschreibt mit welcher Frequenz Symbole übertragen werden, also wie viele Symbole pro
Sekunde (auch Baud-rate genannt) übertragen werden oder übertragen werden können. \\
Die Datenrate beschreibt hingegen, wie viel Information pro Sekunde übertragen werden kann, und ist
gewissermßaen abhängig von der Symbolrate. Ein übertragenes Symbol kann ein Bit sein, es kann aber auch
eindeutig zuordenbar zu mehreren Bits gleichzeitig sein, wodurch eine viel höhere Datenrate erzielt werden kann.

\subsection{Kann durch die verwendung von mehr Symbolen die Datenrate unbegrenzt erhöht werden?}
Die Datenrate würde von der erhöhung der verwendeten Symbolen (wodurch mehr Information codiert werden kann)
natürlich sehr Profitieren. Mit jedem zusätzlichen Symbol kann man vermutlich bis zu ein weiters Bit codieren,
also könnte man durch das verwenden von beliebig vielen Symbolen beliebig hohe Datenraten erzielen. Das Problem
ist ob man die ganzen Symbole noch eindeutig zuordnen kann, da der ganze Vorteil der höheren Symbolrate keine
höhere Datenrate bringt, wenn die Symbole nicht eindeutig identifiziert werden können, wodurch natürlich wieder
Information verloren geht.

\subsection{}
\subsection{}


\section{Aufgabe: Theorem von Nyquist/Shannon}
\subsection{Maximale Datenrate nach Claude Shannon}
Nach Shannon ist die maximale Datenrate $H*log_2(1 + S/N)$ bit/s, Bandweite H und Signalstärke S (Noise/Rauschen N).
Unsere Bandweite H ist $2.2 Mhz - 0.14 Mhz = 2.06 Mhz$, die Signalstärke S ist 10.3616 (10*lg(10.3616) = 45db = SNR = 10 * lg(S/N)). \\
Eingesetzt ergibt dies also dass die maximale Datenrate $2.06 Mhz * log_2(1 + 10.3616) = 7.22...$ Mbit/s.

\subsection{Höchstmöglichste Übertragungsraten mit N-QAM}
Nach Nyquist's Theorem ist die maximal mögliche Datenrate höchstens $2 * H * log_2 V$ bit/s, wobei
H die maximale Frequenz in der Fouriertransformation ist, und V die anzahl der verschiedenen verwendeten Symbole.
Demnach sind es jeweils abhängig von der Anzahl der verwendeten Symbole (V = N), $4.4 * (2/4/6/8)$ Mbit/s = 8.8/17.6/26.4/35.2 Mbit/s.

\subsection{Maximal wirklich erzielte Datenraten}
Weit höhere. Vermutlich durch das verwenden von mehr bzw. Höheren Frequenzen, sowie neuen Techniken wie z.B. Glasfaser und der dafür
sehr Wichtigen Elektrotechnik.


\end{document}


