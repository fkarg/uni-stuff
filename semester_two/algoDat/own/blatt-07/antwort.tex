\documentclass{scrartcl}
\usepackage{german}
\usepackage[utf8]{inputenc}
\usepackage[german]{babel}
\usepackage{amssymb} % what does it do?
\usepackage{graphicx} % I can't do that yet
\usepackage{fancyhdr} % what does it do?
\usepackage{lastpage} % what does it do?
\setlength{\parskip}{\medskipamount} % thats reasonable
\setlength{\parindent}{0pt} % whatever that does


%%%%%%%%%%%%%%%%%%%%%%%%
% Kopf- und Fusszeilen %
%%%%%%%%%%%%%%%%%%%%%%%%
\pagestyle{fancy}
\lhead{
    \begin{tabular}{ll}
        Felix Karg & 4342014\\
    \end{tabular}
}
\chead{Info II - AlgoDat}
\rhead{
    \begin{tabular}{rr}
        \today{} \\
        Seite \thepage{} von \pageref{LastPage}
    \end{tabular}
}
\lfoot{}
\cfoot{}
\rfoot{}

%%%%%%%%%%%%%%%%%%%%%%%%
% Anfang des Dokuments %
%%%%%%%%%%%%%%%%%%%%%%%%
\begin{document}

\section*{Antworten zu Übungsblatt Nr. 7}

\section*{Aufgabe 2}

\begin{verbatim}

Anzahl der Blockoperationen im schlechtesten Fall für reverse,
n (Anzahl gespeicherter Elemente)
B (Blockgröße)
M (Größe des schnellen speichers)

reverse
Worst case: jedes Element der LinkedList ist in einem anderen Block
verteilt über den schnellen speicher, wodurch wir bereits n Blockoperationen haben.
Wenn außerdem unsere LinkedList selber auf einem anderen Block ist brauchen wir für
reverse mindestens n + 1 Blockoperationen.

splice
Worst case: Klassendefinition in anderem Block als Information, somit bereits 2

// _____________________________________________________________________________
LinkedList LinkedList::splice(LinkedListItem* begin,
                              LinkedListItem* end) {
    if (begin != firstItem) {
        begin->previousItem->nextItem = end->nextItem; // plus 3 (worst)
        // ^        ^                    ^  required
    } else {
        firstItem = end->nextItem;
    }
    if (end != lastItem) {
        end->nextItem->previousItem = begin->previousItem; // plus 1 (worst, anderen
                                                           // beiden bereits geladen)
        //        ^                         required, anderen beiden sind bereits geladen
    } else {
        lastItem = begin->previousItem;
    }

    // bereits geladen
    begin->previousItem = NULL;
    end->nextItem = NULL;

    // plus 1
    LinkedList list;

    list.firstItem = begin;
    list.lastItem = end;

    return list;
    // total: 7 Blockoperationen im worst case.
    // meist weniger.
}

\end{verbatim}

Damit wären wir bei $\theta(n)$ für reverse und $\theta(1)$ für splice. \\
(da konstante faktoren)

\end{document}

