\chapter{Introduction}\label{chap:introduction}

Hi-C is a well-known method for getting 3D-interaction information about genomes \cite{wingett2015hicup}. This information, however usually is quite biased (inherently based on the method), and can be corrected through an iterative method \cite{imakaev2012iterative}.


\section{Algorithm}
\todo{describe the algorithm}
\extend{THIS:}
Our fundamental assumption is that every location in our Matrix has in total as
many interactions (with other locations) as every other location.
Taking this in mind, the algorithm itself is pretty straightforward.


\section{Operation}
\draft{Change Name!}
smb can be run on any Unix-based operating system (tested using ubuntu-18.04
and macOS) with Python, Rust and common development packages installed (e.g.
libopenssl-dev, python3-dev, build-essential, ...). For best performance, the
size of the matrix should correlate with the number of available cores and the
amount of available RAM \extend{Give rough factors!}.



\todo{Motivation as to why you'd want to do it at all and like this}
\extend{The motivation being: Rust is fast and still memory-safe. Also can easily be parallelized.}




