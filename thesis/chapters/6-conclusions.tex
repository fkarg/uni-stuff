\chapter{Conclusions}\label{chap:conclusion}


\section{Comparing the Different Implementations}

\paragraph{Memory and Runtime:}
Both Memory needs and Runtime length have been discussed extensively in
\secref{sec:memory} and \secref{sec:runtime} respectively. Comparisons can be
seen in \tabref{tab:compmem} for memory and \figref{fig:runtime} for runtime.
In short: the new ICE implementation in Rust requires a considerable amount of
memory less than the Python implementation, during correction. After the
correction, it peaks at a slightly higher value than the Python requires at any
point in time. KR is in the middle for both, memory-wise. Additionally, the new
implementation requires only about half the runtime length compared to the
Python implementation, beating KR by about 10 minutes in both test cases.


\paragraph{Unused Potential:}\label{sec:potential} This includes the further
optimization of both the C++ and Rust versions, and increasing their level of
integration. Apparently, a revision of the KR implementation is in progress,
needing considerably less memory for bigger matrices. The Rust version could be
revised to make more effective use of parallelizable opportunities, and
integrated better in the Python part to reduce peak memory usage tremendously.


\paragraph{Overall:} each of the current implementations has their own
advantages and disadvantages, providing more flexibility for adapting to
differently constrained resource environments.


\section{Python interacting with Rust:} Even though there have been
difficulties for getting the code packaged, it worked surprisingly well after
this was fixed. There is still work ongoing to make Rust usage from Python
considerably easier. It is well needed, as there is not one standard
approach but several partially similar, but exclusive ones.

The approach recommended by the official Rust Book turned out to be not as easy
as expected regarding packaging. However, two other common approaches make
significant promises regarding this. Thus, it is probably easier to package
Rust code for Python using either one of the two other integration approaches
described in \secref{sec:api}.

\textbf{Overall, more extensive testing for integration approaches is recommended.}


\section{Open questions}

One of the biggest remaining open questions is regarding the more effective use
of multi-core hardware, as to how much it would be possible to actually do
this, in Rust or any other language, and how much of a benefit it would bring.
Both the pre-processing and post-processing mainly use only one core during
their execution. Some of it may not at all be parallizable, but there is
probably several low-hanging fruits regarding optimization.

Other questions are regarding the in \secref{sec:potential} mentioned currently
unused potential, particularly the closer integration from within Python, as it
would directly affect memory requirements. A pure implementation in either C++
or Rust is likely to need less memory than when integrated with Python,
based on the numbers and the reasons for the numbers seen in
\secref{sec:maxmem} about peak memory usage.

Based on available data, the available resources are far from sufficient for
comparing runtimes of considerably bigger matrices (see \secref{sec:specs} for
available resources, and \secref{sec:data} for data about the compared
matrices).
However, with sufficient resources at least this open question could be
answered: For the biggest matrices, is KR still only the second fastest, or
will it actually be faster and more memory efficient than this implementation?














\newpage



\todo{Python and KR implementation: Advantages and Disadvantages for both Implementation and language?}

\draft{use more formal language}

\todo{clean up GitHub and add documentation (on last day)}


\todo{make sure everything is cited appropriately!}

\todo{make sure everything is cited appropriately!}

\todo{make sure everything is cited appropriately!}




% For presentation:

% \todo{for presentation: 1/3 for everyone, 1/3 for supervisor + experts, 1/3 only I am expert}

% \todo{biggest points: Python to rust and how it worked, give code examples but not too much}

