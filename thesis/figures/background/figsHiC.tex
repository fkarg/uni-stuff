\begin{figure}[t]
\begin{centering}
%    \subfloat[Diagram summarising the Hi-C experimental protocol. The red and blue rectangles represent cross-linked restriction fragments while the yellow marker shows the position of biotin incorporation.]
%    \subfloat[Generation of the Hi-C ligation junction sequence by successive digestion (with HindIII in this example), fill in and blunt-ended ligation steps. The modified restriction site sequence is not found in the original genomic sequence.]
    {\includegraphics[scale=4]{figures/background/f1000research-4-7903-g0000.jpg}}
    \caption[Summarised Hi-C protocol]
    {\textbf{a)} Diagram summarising the Hi-C experimental protocol. The red and blue rectangles represent cross-linked restriction fragments while the yellow marker shows the position of biotin incorporation. \textbf{b)} Generation of the Hi-C ligation junction sequence by successive digestion (with HindIII in this example), fill in and blunt-ended ligation steps. The modified restriction site sequence is not found in the original genomic sequence.
    Source: https://www.ncbi.nlm.nih.gov/pmc/articles/PMC4706059/?report=reader (HiCUP-paper)}
    \label{fig:HiC}
\end{centering}
\end{figure}



    % {\includegraphics[scale=0.8]{figures/background/Chromosome_conformation_techniques.jpg}}
    % \caption[Comparison among 3C and its derived methods]{\textbf{Comparison among 3C and its derived methods} Source: https://en.wikipedia.org/wiki/File:Chromosome\_conformation\_techniques.jpg}
