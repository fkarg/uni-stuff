\documentclass{scrartcl}
\usepackage{german}
\usepackage[utf8]{inputenc}
\usepackage[german]{babel}
\usepackage{amssymb} % what does it do?
\usepackage{graphicx} % I can't do that yet
\usepackage{fancyhdr} % what does it do?
\usepackage{lastpage} % for 'lastpage' showing
\setlength{\parskip}{\medskipamount} % thats reasonable
\setlength{\parindent}{0pt} % whatever that does


%%%%%%%%%%%%%%%%%%%%%%%%
% Kopf- und Fusszeilen %
%%%%%%%%%%%%%%%%%%%%%%%%
\pagestyle{fancy}
\lhead{
    \begin{tabular}{ll}
        Felix Karg \\
    \end{tabular}
}
\chead{Systeme I}
\rhead{
    \begin{tabular}{rr}
        \today{} \\
        Seite \thepage{} von \pageref{LastPage}
    \end{tabular}
}
\lfoot{}
\cfoot{}
\rfoot{}

%%%%%%%%%%%%%%%%%%%%%%%%
% Anfang des Dokuments %
%%%%%%%%%%%%%%%%%%%%%%%%
\begin{document}

\section*{Antworten zu Übungsblatt Nr. 9}

\section*{Aufgabe 1}
Ein laut dem Bankier-Algorithmus unsicherer Zustand ist bereits ein solcher,
der zu einem Deadlock führen kann, nicht einer der Zwingend zu einem führt. \\
Bsp für einen unsicheren Zustand: \\
Zwei Ressourcen und Zwei Prozesse, die zum abschließen jeweils zwei Ressourcen
benötigen. Nun hat bereits Prozess1 eine Ressource angefordert und erhalten.
Dieser Zustand wird laut dem Bankier-Algorithmus als Unsicher bewertet, da wenn
nun Prozess2 eine Ressource Anfordert und Erhält, ein Deadlock sicher ist, sollten
die Prozesse die Ressourcen erst nach beendingung ihrer selbst diese wieder freigeben.
Wenn allerdings nun zuerst Prozess1 die Zweite Ressource erhält und abschließen kann,
kann danach Prozess2 Beide Ressourcen erhalt und außerdem abschließen, ohne dass es
zu einem Deadlock geführt hat. \\ \\
(wofür braucht man dafür Pseudo-Code, das wird doch nur noch länger?)

\section*{Aufgabe 2}
\begin{itemize}
    \item[a)] 1: Kein Deadlock möglich, da immer ein Prozess der beiden alle 
        Ressourcen bekommen und nach abschluss diese wieder Freigeben kann,
        Ohne auf den anderen Warten zu müssen.
    \item[a)] 2: Deadlock möglich, genau dann, Wenn alle Drei Prozesse jeweils zwei
        Ressourcen benötigen, und jeweils eine Bereits angefordert und erhalten haben,
        und nun die jeweils zweite anfordern.
    \item[b)] $V \geq n*(M - 1) + 1$, damit ist noch mindestens eine Ressource frei,
        wenn alle Prozesse alle bis auf eine Ressource bekommen haben. Nachdem der
        Prozess der diese bekommen hat abgeschlossen ist, stehen wieder ausreichend
        Ressourcen für andere Prozesse zur verfügung, damit diese auch abschließen können.
    \item[c)] Nur, falls mindestens einer der drei Prozesse nur eine Ressource insgesamt
        benötigt, oder P3 erst nach beendigung von P1 oder P2 eine Ressource entweder
        Anfordert oder Bekommt. Andernfalls ist dieser Zustand nicht sicher
        (Begründung: a2).
\end{itemize}

\section*{Aufgabe 3}
\begin{itemize}
    \item[a1)] SJF: $Z_5, Z_2, Z_1, Z_4, Z_3$, die Durchschnittliche 
        Wartezeit ist minimal.
    \item[a2)] $5 + 3 + 7 + 6 + 1 = ~22$ [min] (ungenauigkeit aufgrund von ... Menschen ...)
    \item[a3)] $(1 + 4 + 9 + 15 + 22) / 5 = 51 / 5 = 10.2 $ [min]

    \item[b1)] \begin{tabular}{l|l|l|l|l|l|l}
            Zeit          & t = 0   & t = 1min  & t = 3min  & t = 6min  & t = 9min  & t = 13min \\
            Kunden(min)   & 5, 6, 7 & 6, 7      & 7         & -         & -         & - \\
            Verkäufer     & 1       & 5         & 5 (3 left)& 7         & 7 (4 left)& - \\
            Azubi         & 3       & 3         & 6         & 6 (3 left)& -         & - \\
        \end{tabular}
    \item[b2)] Nach ~13min.
    \item[b3)] $(1 + 3 + (5 + 1) + (3 + 6) + (6 + 7) ) / 5 = 32 / 5 = 6.4 $ [min]

    \item[c1)] \begin{tabular}{l|l|l|l|l|l|l}
            Zeit          & t = 0   & t = 6min  & t = 7min  & t = 10min & t = 11min \\
            Kunden(min)   & 1, 3, 5 & 1, 3      & 1         & -         & -         \\
            Verkäufer     & 6       & 5         & 5 (4 left)& 5 (1 left)& -         \\
            Azubi         & 7       & 7 (1 left)& 3         & 1         & -         \\
                \end{tabular}
    \item[c2)] Nach ~11min
    \item[c3)] $(7 + 6 + (6 + 5) + (7 + 3) + (7 + 3 + 1) ) / 5 = (13 + 11 + 10 + 11) / 5
        = 45 / 5 = 9.0 $ [min] \\ \\
        Aus sicht des Azubis ist besser Strategie c, da er dann 2min eher (bzw. länger)
        Pause machen kann. Für den Kunden ist allerdings bei weitem Strategie b Besser,
        um knapp das 1.5-fache (9 / 6.4 = 1.40625), da die Wartezeit im Durchschnitt
        um einiges Kürzer ist.

\end{itemize}

\section*{Aufgabe 4}
Wenn ein Prozess zwischenreinkommt, der eine Kürzere Durchlaufzeit hat, als der momentan
ausgeführte noch übrig hat. Wenn die Zeit die der momentan ausführende noch übrig hat
sehr groß, und die Durchlaufzeit des neu dazukommenden sehr kurz ist, wird der unterschied
gut sichtbar.


\end{document}


Beispiel für Text, der aus einem Terminal kopiert wurde:

\begin{verbatim}
osswald@tfpool17 / $ df -h
Filesystem                           Size  Used Avail Use% Mounted on
/dev/sda4                            375G   41G  316G  12% /
dev                                  3.9G     0  3.9G   0% /dev
run                                  3.9G  480K  3.9G   1% /run
tmpfs                                3.9G     0  3.9G   0% /dev/shm
\end{verbatim}

Aufzählungen sind mit \verb_enumerate_ möglich:
\begin{enumerate}
\item Erster Punkt
\item Zweiter Punkt
\item Dritter Punkt
\end{enumerate}

\subsection*{Aufgabe 2}

Mathematische Formeln:
\begin{equation}\label{gauss}
    \sum_{i=1}^{n} i = \frac{n(n+1)}{2}
\end{equation}


Formel \ref{gauss} wird auch \emph{Gaußsche Summenformel} genannt.

Formeln können auch im Text eingebunden werden, z.B. $E = mc^{2}$.

\subsection*{Aufgabe 3}

Tabellen können mit \verb_tabular_-Umgebungen eingegeben werden:

\begin{center}
\begin{tabular}{l|l|l}
Datei             & Dateirechte        & Größe   \\
\hline
dokument.txt      & \verb_-rw-r--r--_  & 300 KB  \\
programm.exe      & \verb_-rwxr-x---_  & 450 KB  \\
mein\_verzeichnis & \verb_drwxr-xr-x_  & ---     \\
\end{tabular}
\end{center}

\end{document}


