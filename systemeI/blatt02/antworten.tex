\documentclass{scrartcl}
\usepackage{german}
\usepackage[utf8]{inputenc}
\usepackage[german]{babel}
\usepackage{amssymb} % what does it do?
\usepackage{graphicx} % I can't do that yet
\usepackage{fancyhdr} % what does it do?
\usepackage{lastpage} % what does it do?
\setlength{\parskip}{\medskipamount} % thats reasonable
\setlength{\parindent}{0pt} % whatever that does


%%%%%%%%%%%%%%%%%%%%%%%%
% Kopf- und Fusszeilen %
%%%%%%%%%%%%%%%%%%%%%%%%
\pagestyle{fancy}
\lhead{
    \begin{tabular}{ll}
        Felix Karg \\
    \end{tabular}
}
\chead{Systeme I}
\rhead{
    \begin{tabular}{rr}
        \today{} \\
        Seite \thepage{} von \pageref{LastPage}
    \end{tabular}
}
\lfoot{}
\cfoot{}
\rfoot{}

%%%%%%%%%%%%%%%%%%%%%%%%
% Anfang des Dokuments %
%%%%%%%%%%%%%%%%%%%%%%%%
\begin{document}

\section*{Antworten zu Übungsblatt Nr. 2}

\subsection*{Aufgabe 1}
\begin{itemize}
  \item[a)] das \verb$d$ steht dafür dass '\verb$meine_dateien$' ein Verzeichnis ist,
    die folgenden \verb$rwx$ dafür dass der Owner alles darf, die folgenden \verb!r-x! dafür dass
    andere Aus dieser Gruppe Lesen und Ausführen (= im Verzeichnis dinge ändern) dürfen,
    und das letzte \verb$--x$ bedeuted dass alle das Verzeichnis Löschen/Bearbeiten
    aber nicht lesen oder schreiben dürfen.
  \item[b)]
\begin{verbatim}
bez@bez-home ~> ssh fk265@login.uni-freiburg.de
fk265@login.uni-freiburg.de's password:
Last login: Tue Nov  1 20:16:06 2016 from -----redacted------
Have a lot of fun...
Linux login1 4.1.34-33-default x86_64
$ cd /tmp
$ vi begruessung_fk265.sh
$ la begruessung_fk265.sh
-rw-r--r-- 1 fk265 uni 84 Nov  1 20:37 begruessung_fk265.sh
$ ./begruessung_fk265.sh
-bash: ./begruessung_fk265.sh: Permission denied
$ chmod u+x begruessung_fk265.sh
$ la begruessung_fk265.sh
-rwxr--r-- 1 fk265 uni 84 Nov  1 20:42 begruessung_fk265.sh
$ ./begruessung_fk265.sh
Bitte geben Sie Ihren Namen ein: Felix
Hallo Felix.
$
\end{verbatim}
  \item[c)]
\begin{verbatim}
$ cd /tmp
$ cp /usr/bin/whoami werbinich_fk265
$ ./werbinich_fk265
fk265
$ echo $NAME

$ export NAME="fk265"
$ echo $NAME
fk265
$ ls -la /usr/bin/whoami
-rwxr-xr-x 1 root root 27360 Sep 29  2015 /usr/bin/whoami
$ ls -la werbinich_$NAME
-rwxr-xr-x 1 fk265 uni 27360 Nov  1 20:46 werbinich_fk265
$ # Die Berechtigungen sind gleich - das war nicht das erwünschte oder?
$ # Alle der Gruppe uni dürfen die Datei bereits ausführen, ihnen wird ihr jeweiliger Benutzername angezeigt.
$ chmod u+s werbinich_fk265
$ ls -la werbinich_$NAME
-rwsr-xr-x 1 fk265 uni 27360 Nov  1 20:46 werbinich_fk265
$
$\end{verbatim}
  \item[d)]
\begin{verbatim}
$ cd /tmp
$ mkdir mein_verzeichnis_$NAME
$ touch mein_verzeichnis_$NAME/loesche_mich.txt
$ chmod a=r mein_verzeichnis_$NAME/loesche_mich.txt
$ chmod a=r mein_verzeichnis_$NAME
$
$ rm mein_verzeichnis_$NAME/loesche_mich.txt
rm: cannot remove ‘mein_verzeichnis_fk265/loesche_mich.txt’: Permission denied
$ la mein_verzeichnis_$NAME
ls: cannot access mein_verzeichnis_fk265/.: Permission denied
ls: cannot access mein_verzeichnis_fk265/..: Permission denied
ls: cannot access mein_verzeichnis_fk265/loesche_mich.txt: Permission denied
total 0
d????????? ? ? ? ?            ? .
d????????? ? ? ? ?            ? ..
-????????? ? ? ? ?            ? loesche_mich.txt
$ chmod u=rwx mein_verzeichnis_$NAME/
$ la mein_verzeichnis_$NAME
total 0
drwxr--r-- 1 fk265 uni     32 Nov  1 21:02 .
drwxrwxrwt 1 root  root 15896 Nov  1 21:07 ..
-r--r--r-- 1 fk265 uni      0 Nov  1 21:02 loesche_mich.txt
$ rm mein_verzeichnis_$NAME/loesche_mich.txt
rm: remove write-protected regular empty file ‘mein_verzeichnis_fk265/loesche_mich.txt’? y
$ la mein_verzeichnis_$NAME
total 0
drwxr--r-- 1 fk265 uni      0 Nov  1 21:08 .
drwxrwxrwt 1 root  root 15896 Nov  1 21:07 ..
\end{verbatim}
  \item[e)]
\begin{verbatim}
$ echo "You can." > $NAME\_readmeifyoucan.txt
$ cat $NAME\_readmeifyoucan.txt
You can.
$ chmod u-rwx $NAME\_readmeifyoucan.txt
$ cat $NAME\_readmeifyoucan.txt
cat: fk265_readmeifyoucan.txt: Permission denied
$ # Ja die Ausgabe würde sich unterscheiden, da benutzer der 
\end{verbatim}
  \item[f)]
\begin{verbatim}
$ cd
$ mkdir systeme-public
$ chmod og+x .
$ # Ausreichend? TODO: Could not test ...
\end{verbatim}

\end{itemize}

\subsection*{Aufgabe 3}
\begin{itemize}
\item[a)] Interne Fragmentierung; Bei manchen Dateisystemen möglicherweise schon, aber eigentlich nicht.
\item[b)] 'Schreiben Sie die neue Belegung auf.' ist es nicht wert in \LaTeX eine Stunde zu investieren. \\ \\
Ein Block ist noch komplett frei, und drei sind als 'gelöscht' markiert, bzw es existiert keine
referenz mehr auf sie. Ja es ist möglich eine 100 KiB datei zu erstellen,
diese wird auf die ersten drei und entweder in das ende von 3 bzw. 4 geschrieben,
 oder belegt alleine die letzte freie Zelle.

\end{itemize}
\end{document}

\section*{Lösungen zu Übungsblatt Nr. 0}

\subsection*{Aufgabe 1}

Beispiel für Text, der aus einem Terminal kopiert wurde:

\begin{verbatim}
osswald@tfpool17 / $ df -h
Filesystem                           Size  Used Avail Use% Mounted on
/dev/sda4                            375G   41G  316G  12% /
dev                                  3.9G     0  3.9G   0% /dev
run                                  3.9G  480K  3.9G   1% /run
tmpfs                                3.9G     0  3.9G   0% /dev/shm
\end{verbatim}

Aufzählungen sind mit \verb_enumerate_ möglich:
\begin{enumerate}
\item Erster Punkt
\item Zweiter Punkt
\item Dritter Punkt
\end{enumerate}

\subsection*{Aufgabe 2}

Mathematische Formeln:
\begin{equation}\label{gauss}
    \sum_{i=1}^{n} i = \frac{n(n+1)}{2}
\end{equation}


Formel \ref{gauss} wird auch \emph{Gaußsche Summenformel} genannt.

Formeln können auch im Text eingebunden werden, z.B. $E = mc^{2}$.

\subsection*{Aufgabe 3}

Tabellen können mit \verb_tabular_-Umgebungen eingegeben werden:

\begin{center}
\begin{tabular}{l|l|l}
Datei             & Dateirechte        & Größe   \\
\hline
dokument.txt      & \verb_-rw-r--r--_  & 300 KB  \\
programm.exe      & \verb_-rwxr-x---_  & 450 KB  \\
mein\_verzeichnis & \verb_drwxr-xr-x_  & ---     \\
\end{tabular}
\end{center}

\end{document}


