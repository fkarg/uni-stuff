\documentclass{scrartcl}
\usepackage{german}
\usepackage[utf8]{inputenc}
\usepackage[german]{babel}
\usepackage{amssymb}  % advanced mathematical symbols
\usepackage{graphicx} % using graphics
\usepackage{fancyhdr} % for the head of the page
\usepackage{lastpage} % makes page numbers work
\setlength{\parskip}{\medskipamount} % thats reasonable
\setlength{\parindent}{0pt}

\usepackage{wrapfig}


%%%%%%%%%%%%%%%%%%%%%%%%
% Kopf- und Fusszeilen %
%%%%%%%%%%%%%%%%%%%%%%%%
\pagestyle{fancy}
\lhead{
    \begin{tabular}{ll}
        Felix Karg & 4342014\\
    \end{tabular}
}
\chead{Review of Efficient representation of ...}
\rhead{
    \begin{tabular}{rr}
        \today{} \\
        Seite \thepage{} von \pageref{LastPage}
    \end{tabular}
}
\lfoot{}
\cfoot{}
\rfoot{}

\title{Review of Efficient representation of finitie sets}
\author{Felix Karg}

%%%%%%%%%%%%%%%%%%%%%%%%
% Anfang des Dokuments %
%%%%%%%%%%%%%%%%%%%%%%%%
\begin{document}
\maketitle

% Review-Template
% 
% Short Overview
% Structure of the talk
% aspects of the topic you present or ignore, and why
% examples for the talk, is there sth running?
% formal and informal definitions and why
% notation
% theorems, proofs
% 
% easily understandable
% allows for someone else to follow through and roughly see the talk
% in-depth-introduction to topic

\section*{Overview + Structure of the talk + Basics}
The Abstract (Overview) is quite informing, and the first paragraph was quite
fluently readable, I would have preferred for the rest to have been in a
similiar matter, or at least easier readable, with a less high density of
definitions and the like. The Definition 2.1 and the following thoughts are
quite peculiar, additionally that we are going to have an automaton with
infinitely many states and still no initial state, but I guess it only makes
sense. What I was wondering about though was how this automaton would be used
then. In retrospect I don't know if I would be calling it an automaton (since
it does something else as far as I understood), but then again it is the
closest to anything else I guess. I did not exactly understand how to interpret
Proposition 2.1 yet, this might change in due time, still, I'm writing this
down as confusion at least existed. I now understood it, and the confusion
arose as to why the master automaton would be recognising languages in
particular, since it has not been defined or even mentioned yet. Then again, it
probably really makes sense to call it an automaton after all. During the talk
I would probably mention how that automaton would be used when introducing it,
or directly after. When introducing the make-procedure, T has not been
mentioned yet, but I can infer that it probably is something like a relational
structure, apparently with tree-like structure. The following Example and table
were confusing at first, since the make(3, 2) is apparently to be interpreted
as 'The state from which 'a' will get me to 3, and 'b' to 2'. I was quite
confused at first since I expected it to be 6 at first, and now I'm convinced
the 8 in there is a mistake as well, as it is probably supposed to be a 9. The
table was confusing at first as well, since the meaningn of the zero did not
get introduced and I would have expected there to be $q_\emptyset$ at first.
Additionally the b-successor of 4 is apparently not 0 but 1.

\section*{Operation on fixed-length languages}
About the $G(q_1, q_2)$-thing in the inter-algorithm: G did not at all get
explained anywhere as of yet. I would have preferred you to describe what inter
does more in detail, or explain the code in detail, otherwise don't show it,
it's only confusing to read code using global variables all over and just about
no comments. 0 in it's special meaning has yet to be introduced, and the
description is useful, but should explain more than leaving questionmarks
behind after reading it the first time.

\section*{Decision Diagrams}
The explanation just before Definition 7.12 may be having a type-error, as the
DD is supposed to be deterministic but $\delta(q, a\Sigma^k)$ retuns the set
$\{q'\}$. Why a set? Is that correct? Why? After the Definition 7.12, I'm
confused. I thought that there would be more of an explanation as to what the
kernel actually is supposed to be, because I do not claim to have understood it
after reading through it three times now. Additionally, afterwards there's
withaout any explanation a master-decision-diagram, the kinter diagram which
apparently has something to do with kernels ... maybe, again using stuff not at
all explained, and two probably regarding to the decision-diagram updated
graphs from the inter-algorithm, probably somehow relatet to kinter ....
honestly, this kind of ending is weird.

\section*{short thoughts}
Guess I failed as a DAU (Dümmster Anzunehmender User) regarding the reading of
this article, I understood quite a bit of it. Then again, I would have
preferred it to be more in detail with some edge-cases to gain a deeper
understanding (but maybe this would have only been my personal preference and a
better explanation would do the trick).

\section*{Conclusion}
Everything was more or less understandable, but it could definitely be
understandable easier and be more concise. The Overview was useful. I think the
comparison to Decision Diagrams mentioned in 'Structure of the talk' did not
exactly happen, but hopefully it will in the following talk, since I guess it
would be interesting as well. There are quite a few formal definitions, I don't
know how useful this is in the actual talk, but having too many formal
definitions is not going to be easy. The notation should be explained more, but
other than that it is quite understandable.


\end{document}
