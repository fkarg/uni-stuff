\documentclass{scrartcl}
\usepackage{german}
\usepackage[utf8]{inputenc}
\usepackage[german]{babel}
\usepackage{amssymb} % what does it do?
\usepackage{graphicx} % I can't do that yet
\usepackage{fancyhdr} % what does it do?
\usepackage{lastpage} % what does it do?
\setlength{\parskip}{\medskipamount} % thats reasonable
\setlength{\parindent}{0pt} % whatever that does
\usepackage[bottom=10em]{geometry}
\usepackage{hyperref}


%%%%%%%%%%%%%%%%%%%%%%%%
% Kopf- und Fusszeilen %
%%%%%%%%%%%%%%%%%%%%%%%%
\pagestyle{fancy}
\lhead{
    Felix Karg
}
\chead{Persönliches Engagement}
\rhead{
    \begin{tabular}{rr}
        \today{} \\
        Seite \thepage{} von \pageref{LastPage}
    \end{tabular}
}
\lfoot{}
\cfoot{}
\rfoot{}

%%%%%%%%%%%%%%%%%%%%%%%%
% Anfang des Dokuments %
%%%%%%%%%%%%%%%%%%%%%%%%
\begin{document}


\section*{Engagement von meiner Seite}
Dies ist ein unvollständiges und informales Dokument mit einer Sammlung von Tätigkeiten
die ich Ehrenamtlich ausgeführt habe. Diese Liste ist bei weitem nicht, wie es im Englischen so
schön heißt, 'exhaustive', also erschöpfend, was hoffentlich erkenntlich wird.

Ich helfe immer wieder und auch immer wieder Gerne. Ich war bereits in der sechsten Klasse im
Büchereiclub, und war dort für zwei Jahre mit zwei anderen dort um den Ausleihenden zu helfen,
mich darum zu kümmern dass alles wieder rechtzeitig zurückkommt oder einfach nur dort zu sein.

Viel geholfen habe ich auch bei den örtlichen Ministranten, die erstan Jahre natürlich nur
Teilnehmend bei Veranstaltungen oder eben Ministrierend, später allerdings auch als ein
mit-Planer der verschiedenen Veranstaltungen, was bis zu 3h/Woche war, allerdings war natürlich
nicht immer etwas zu Planen.

Als ich irgendwann das OpenLab entdeckt habe, das ein Ehrenamtlicher
Verein von Technikinteressierten in Augsburg ist, und dort immer häufiger zugegen war, habe ich
natürlich auch mehr und mehr dort ausgeholfen, sei es neue Getränke holen oder zu Putzen, irgendwie
die Nächste Veranstaltung (für 10 bis zu 80 Leute) mitplanen oder mitgestalten, oder einfach nur
Leuten die kamen und sich Hilfe erhofften Hilfe anzubieten. Ich hatte versucht relativ regelmäßig
dort zu sein, und in meinen letzten Schuljahren hat das auch je nach Interpretation gut geklappt.
Es gab Monate in denen ich ca. 4 mal pro Woche dort war, allerdings auch solche in denen ich gerade ein
mal pro Woche anwesend war. Jeweils natürlich für mehrere Stunden, und natürlich war nicht immer
etwas zu tun, aber es hat selten jemand Buch geführt wer wie viel getan hat.

Jedes mal wenn ich zu einem Chaos-Event fahre (also z.B. demnächst wieder die GPN, \url{gulas.ch}),
helfe ich mit wo ich kann. Diese Veranstaltungen funktionieren insgesamt auch nur, weil es so viele
Ehrenamtliche Helfer gibt, die Unglaublich viel Organisieren und mithelfen. Solch ein Event hat
im Normalfall 4 Tage, allerdings muss natürlich auch davor und danach viel auf und wieder abgebaut
werden. Leider habe ich selten die möglichkeit auch vor oder nach dem Event selbst anwesend zu sein,
aber wenn es mir Möglich ist helfe ich natürlich mit. Während der Veranstaltung kann man selbst
entscheiden wann und wo man helfen möchte, ich Helfe im Durchschnitt um die 20 Stunden, möglicherweise
auch mehr. Ich war bisher auf mindestens sechs solcher Events.

Auf dem Haus auf dem ich momentan wohne (12 Zimmerbewohner momentan) veranstalten wir auch Häufiger
zum Beispiel gemeinsame Essen, fahrten oder auch Wanderungen zu Burgen oder einfach nur durch die
Frische Luft, oder Vorträge die im Normalfall für alle sehr Interessant sind. Momentan bin ich einer
der Haupt-Organisatoren bei solchen Veranstaltungen, was natürlich auch Ehrenamtlich ist.
Durchschnittlich ist der Aufwand mindestens 2h/Woche, wo manchmal ganze Abende zusätzlich nötig sind
um allein die Veranstaltungen der Nächsten Woche Organisiert zu bekommen.


Das sind ein paar der Dinge die mir spontan einfallen, aber ich helfe natürlich wo ich kann.

% EH: ~15h, 1x
% GPN: ~20h, 2x
% Congress: ~20h, 2x
% Camp: ~30h, 1x

% Tutoring Englisch und Mathe: wenn nötig und bei scheinbarer Anstrengung: Zusätzliche Stunden.
% (Informatik-) Lehrer im Unterricht: immer wieder
% Ministranten: Anfangs eben Ministrieren, später auch mitorganisation von Veranstaltungen, bis zu 3h/Woche.
% Viel Eherenamtliche arbeit im OpenLab, zum aufrechterhalten der Vereinsinfrastruktur oder saubermachen.
% Dort: Helfen und geholfen werden; Leute mit Problemen Betreuen und Leiten.





\end{document}


