\documentclass{scrartcl}
\usepackage{german}
\usepackage[utf8]{inputenc}
\usepackage[german]{babel}
\usepackage{amssymb} % what does it do?
\usepackage{graphicx} % I can't do that yet
\usepackage{fancyhdr} % what does it do?
\usepackage{lastpage} % what does it do?
\setlength{\parskip}{\medskipamount} % thats reasonable
\setlength{\parindent}{0pt} % whatever that does
\usepackage[bottom=10em]{geometry}
\usepackage{hyperref}
\usepackage{lmodern}


%%%%%%%%%%%%%%%%%%%%%%%%
% Kopf- und Fusszeilen %
%%%%%%%%%%%%%%%%%%%%%%%%
\pagestyle{fancy}
\lhead{
    Felix Karg
}
\chead{Persönliches Engagement}
\rhead{
    \begin{tabular}{rr}
        \today{} \\
        Seite \thepage{} von \pageref{LastPage}
    \end{tabular}
}
\lfoot{}
\cfoot{}
\rfoot{}

%%%%%%%%%%%%%%%%%%%%%%%%
% Anfang des Dokuments %
%%%%%%%%%%%%%%%%%%%%%%%%
\begin{document}


\section*{Engagement von meiner Seite}
Dies ist ein unvollständiges und informales Dokument mit einer Sammlung von Tätigkeiten,
die ich ehrenamtlich ausgeführt habe. Diese Liste ist bei weitem nicht, wie es im Englischen so
schön heißt, „exhaustive“, also erschöpfend, was hoffentlich erkenntlich wird.

Ich helfe immer wieder und auch immer wieder gerne. Ich war bereits in der sechsten Klasse im
Büchereiteam, und war dort für zwei Jahre mit zwei weiteren dort um den Ausleihenden zu helfen
und mich darum zu kümmern dass alles wieder rechtzeitig zurückkommt.

Viel geholfen habe ich auch bei den örtlichen Ministranten, die ersten Jahre natürlich nur
teilnehmend bei Veranstaltungen oder eben ministrierend, später allerdings auch als ein
Mitplaner der verschiedenen Veranstaltungen, was bis zu 3 stunden pro Woche in anspruch nahm.

Als ich irgendwann das OpenLab entdeckt habe, das ein ehrenamtlicher
Verein von Technikinteressierten in Augsburg ist, und dort immer häufiger zugegen war, habe ich
natürlich auch mehr und mehr dort ausgeholfen, sei es beim Getränke holen oder beim Putzen, irgendwie
die nächste Veranstaltung (für 10 bis zu 80 Leute) mitzuplanen oder mitzugestalten, oder einfach nur
Leuten Hilfe anzubieten. Ich habe versucht relativ regelmäßig
dort zu sein, und in meinen letzten Schuljahren hat das auch je nach Interpretation gut geklappt.
Es gab Monate in denen ich ca. vier mal pro Woche dort war, allerdings auch solche in denen ich gerade ein
mal pro Woche anwesend war, jeweils natürlich für mehrere Stunden.

Jedes mal wenn ich zu einem Chaos-Event fahre (also z.B. demnächst wieder die GPN, \url{gulas.ch}),
helfe ich mit wo ich kann. Diese Veranstaltungen basieren auf dem Prinzip dass ehrenamtliche Helfer
organisieren und mithelfen sowie beim Auf- und Abbau mithelfen. So oft es mir möglich ist engagiere
ich mich bei diesen Events und helfe während der Dauer von meist vier Tagen im Durchschnitt um die
20 Stunden. Bisher war ich auf mindestens sechs dieser Events.

In der WG, in welcher ich momentan wohne (diese beheimatet momentan 12 Bewohner) veranstalten wir auch des öfteren
beispielsweise gemeinsame Essen, Fahrten und auch Wanderungen. Des weiteren gibt es Vorträge, die im Normalfall
für alle sehr interessant sind. Momentan bin ich einer
der Haupt-Organisatoren bei diesen Veranstaltungen, natürlich auf ehrenamtlicher Basis.
Durchschnittlich beträgt dieser Aufwand mindestens zwei Stunden pro Woche, wobei manchmal ganze Abende
für die Organisation der Veranstaltungen der nächsten Woche nötig sind.

% Das sind ein paar der Dinge die mir spontan einfallen, aber ich helfe natürlich wo ich kann.

% EH: ~15h, 1x
% GPN: ~20h, 2x
% Congress: ~20h, 2x
% Camp: ~30h, 1x

% Tutoring Englisch und Mathe: wenn nötig und bei scheinbarer Anstrengung: Zusätzliche Stunden.
% (Informatik-) Lehrer im Unterricht: immer wieder
% Ministranten: Anfangs eben Ministrieren, später auch mitorganisation von Veranstaltungen, bis zu 3h/Woche.
% Viel Eherenamtliche arbeit im OpenLab, zum aufrechterhalten der Vereinsinfrastruktur oder saubermachen.
% Dort: Helfen und geholfen werden; Leute mit Problemen Betreuen und Leiten.





\end{document}


