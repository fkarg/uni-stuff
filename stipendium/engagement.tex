\documentclass{scrartcl}
\usepackage{german}
\usepackage[utf8]{inputenc}
\usepackage[german]{babel}
\usepackage{amssymb} % what does it do?
\usepackage{graphicx} % I can't do that yet
\usepackage{fancyhdr} % what does it do?
\usepackage{lastpage} % what does it do?
\setlength{\parskip}{\medskipamount} % thats reasonable
\setlength{\parindent}{0pt} % whatever that does
\usepackage[bottom=10em]{geometry}


%%%%%%%%%%%%%%%%%%%%%%%%
% Kopf- und Fusszeilen %
%%%%%%%%%%%%%%%%%%%%%%%%
\pagestyle{fancy}
\lhead{
    Felix Karg
}
\chead{Persönliches Engagement}
\rhead{
    \begin{tabular}{rr}
        \today{} \\
        Seite \thepage{} von \pageref{LastPage}
    \end{tabular}
}
\lfoot{}
\cfoot{}
\rfoot{}

%%%%%%%%%%%%%%%%%%%%%%%%
% Anfang des Dokuments %
%%%%%%%%%%%%%%%%%%%%%%%%
\begin{document}


\section*{Engagement von meiner Seite}
Hallo, dies ist ein unvollständiges und informales Dokument mit einer Sammlung von Tätigkeiten
die ich Ehrenamtlich ausgeführt habe. Diese Liste ist bei weitem nicht, wie es im Englischen so
schön heißt, 'exhaustive', also erschöpfend, was hoffentlich erkenntlich wird.

Ich helfe immer wieder und auch immer wieder Gerne. Ich war bereits in der sechsten Klasse im
Büchereiklub, und war dort für Zwei Jahre mit zwei anderen dort um den Ausleihenden zu helfen,
mich darum zu kümmern dass alles wieder rechtzeitig zurückkommt oder einfach nur dort zu sein.

Viel geholfen habe ich auch bei den Örtlichen Ministranten, die erstan Jahre natürlich nur
Teilnehmend bei Veranstaltungen oder eben Ministrierend, später allerdings auch als ein
mit-Planer der verschiedenen Veranstaltungen, was bis zu 3h/Woche war, allerdings war natürlich
nicht immer etwas zu Planen.

Als ich irgendwann das OpenLab entdeckt habe, das ein Ehrenamtlicher
Verein von Technikinteressierten in Augsburg ist, und dort immer häufiger zugegen war, habe ich
natürlich auch mehr und mehr dort ausgeholfen, sei es neue Getränke holen oder zu Putzen, irgendwie
die Nächste Veranstaltung (für bis zu 80 Leute) mitplanen oder auch mitgestalten, oder einfach nur
Leuten die kamen und sich Hilfe erhofften Hilfe anzubieten. Ich hatte versucht relativ regelmäßig
dort zu sein, und in meinen letzten Schuljahren hat das auch je nach interpretation gut geklappt.
Es gab Monate in denen ich ca. 4/Woche dort war, allerdings auch solche in denen ich gerade ein
mal pro Woche anwesend war. Jeweils natürlich für mehrere Stunden, und natürlich war nicht immer
etwas zu tun, aber es hat selten jemand Buch geführt wer wie viel getan hat.

Als ich damals in
der Achten und Neunten Klasse Tutor für Mathe bzw. Englisch war, habe ich auch manchen Schülern
angeboten, Kostenlos zusätzliche Stunden zu unterrichten, wenn ein Test bevorstand und derjenige
Schüler sich wirklich angestrengt oder Mitgearbeitet hat, also nicht die ganze Zeit demotiviert war.





EH: ~15h, 1x
GPN: ~20h, 2x
Congress: ~20h, 2x
Camp: ~20h, 1x


Tutoring Englisch und Mathe: wenn nötig und bei scheinbarer Anstrengung: Zusätzliche Stunden.
(Informatik-) Lehrer im Unterricht: immer wieder
Ministranten: Anfangs eben Ministrieren, später auch mitorganisation von Veranstaltungen, bis zu 3h/Woche.
Viel Eherenamtliche arbeit im OpenLab, zum aufrechterhalten der Vereinsinfrastruktur oder saubermachen.
Dort: Helfen und geholfen werden; Leute mit Problemen Betreuen und Leiten.





\end{document}


