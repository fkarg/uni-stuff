\documentclass{scrartcl}
% \usepackage{german}
\usepackage[utf8]{inputenc}
\usepackage[german]{babel}
% \usepackage{amssymb} % what does it do?
% \usepackage{graphicx} % I can't do that yet
\usepackage{fancyhdr} % what does it do?
\usepackage{lastpage} % what does it do?
\setlength{\parskip}{\medskipamount} % thats reasonable
\setlength{\parindent}{0pt} % whatever that does
\usepackage{hyperref}
\usepackage[bottom=10em]{geometry}



%%%%%%%%%%%%%%%%%%%%%%%%
% Kopf- und Fusszeilen %
%%%%%%%%%%%%%%%%%%%%%%%%
\pagestyle{fancy}
\lhead{
    Felix Karg
}
\chead{Persönliche Motivation}
\rhead{
    \begin{tabular}{rr}
        \today{} \\
        Seite \thepage{} von \pageref{LastPage}
    \end{tabular}
}
\lfoot{}
\cfoot{}
\rfoot{}

%%%%%%%%%%%%%%%%%%%%%%%%
% Anfang des Dokuments %
%%%%%%%%%%%%%%%%%%%%%%%%
\begin{document}


\section*{Motivation für die Bewerbung}

Hallo an wer auch immer das Lesen wird. \\


\subsection*{Allgemeines Geplänkel}

Da es weder vorgaben für Inhalt, Umfang oder Format gibt, dachte ich mir dass ich das ganze
im Stil eines 'Stream of Consciousness' verfasse, also einfach Niederschreibe was mir gerade
durch den Kopf geht. Ich vermute dass das eine Möglichkeit ist einen relativ Guten einblick
in die Gedankenwelt einer anderen Person zu bekommen, auch wenn natürlich nicht alles in der
nötigen Geschwindigkeit aufgeschrieben werden kann, Bzw. natürlich die Gedanken selten dermaßen
Konkret formuliert sind.  \\


\subsection*{Informationen über mich}

Ich bin im Prinzip ein ganz normaler Informatik-Student an der Universität in Freiburg. Das
stimmmt eventuell nicht so ganz, aber das möchte ich nicht vorwegnehmen. Ich interessiere mich
ehrlichgesagt für alles mögliche. Angefangen bei Physik und Mathematik über Allgemeines
Planungsdenken, Philosophie, Chemie, Weltraum und Astronomie, natürlich Informatik, und
so viele teilweise sehr spezielle Bereiche in jedem der Bereiche. Außerdem mag ich Shakespeare,
aber nur auf Englisch, die Deutschen Versionen kommen nicht im Ansatz an das Sprachspiel hin.
Als ich Spontan meine Freundin fragte was für dinge ich mögen würde kamen folgende Antworten:
Programmieren, Informatik, KI, Mathematik, Strategie und Problemlösung, Eigenständiges Denken und Lernen,
Judo, Cookies, Sudokus, Bücher, Raketen. Die Reihnfolge ist möglicherweise ein Wenig durcheinander
aber im großen und ganzen fehlt da momentan nur die Physik, über die wir uns leider viel zu wenig
unterhalten, was daran liegen mag dass wir uns viel über Mathematik oder anderes Unterhalten.
Jedenfalls würde ich mich selbst als Weltoffen beschreiben, Reise im allgemeinen gerne
und Diskutiere gerne, oder anders gesagt: führe gerne Interessante, Anregende Gespräche über
alle möglichen Themen. Also was ich mir gewissermaßen von einem Stidendium erhoffe ist
mehr Leute zu treffen die auf ähnliches lust haben, und dass sich mehr möglichkeiten bieten
solche Interessanten Gespräche zu führen. Allgemeiner gesprochen noch: ich erhoffe mir
gewissermaßen Gleichgesinnte zu treffen, was nicht einfach ist. Andere Weltoffene Leute,
oder solche die einfach nur Interesse an irgendwas haben, und begeistert sind das zu tun. \\

\subsection*{Erwartungen an ein Stipendium}

Außerdem glaube ich dass es sehr viel spaß machen würde Dinge mit solchen Leuten zu unternehmen,
also gemeinsam Vorträge zu besuchen, oder sich auch allgemein wenn man in der Stadt ist zu besuchen
oder sich auszutauschen. Ein anderer Großer Punkt ist die Persönliche Weiterbildung. Ich versuche
mich ja ständig irgendwie weiterzubilden, zu verbessern. Wer nicht Vorwärtsgeht geht rückwärts.
Wer keinen Vortschritt macht macht Rückschritte. Im Deutschen klingt das irgendwie doof. Apropos:
Mir ist im prinzip egal ob jemand mit mir Deutsch oder Englisch redet, ob ich etwas auf Deutsch
oder auf Englisch höre oder sehe. Englisch ist die Informatiker-Sprache, und wer sie nicht
beherrscht hat keine möglichkeit sich mit jemandem z.B. aus Amerika, England, oder auch Dänemark
oder Japan zu unterhalten, obwohl derjenige eindeutig ein Interessanter Gesprächspartner wäre.
Es kam auch nur irgendwie mit der Zeit, aber wenn man sich häufiger einfach mit solchen Leuten
unterhält wird das ganze schnell einfacher. Ich komme ab. Das ist jedenfalls auch einer der
Weiterbildungspunkte; In dem fall würde es einen schon an der Kommunikation behindern. Ich möchte
allerdings nicht nur Wissen ansammeln, sondern es auch wieder weitergeben, auch wenn ich momentan
wenig Wissen besitze das andere irgendwie interessiert. Wenn möglich helfe ich natürlich meinen
Studienkollegen, und aufgrund meiner bereits etwas größeren Erfahrung in der Informatik bin ich
sogar einer der Ansprechpartner wenn es größere Probleme gibt; Und natürlich helfe ich wo ich kann. \\

\subsection*{Interessen}

Wie bereits erwähnt Interessiere ich mich sehr für verschiedene Gebiete, und das hat einen Grund:
Ich möchte irgendwie die Welt verstehen. Dafür ist nicht nur wichtig was momentan an Forschung
passiert, sondern auch welche großartigen Durchbrüche bereits in der Vergangenheit geschehen sind.
Ein Beispiel dafür wäre die Allgemeine Relativitätstheorie. Natürlich, sie ist bereits über 100
Jahre alt, aber ist sie deswegen weniger Relevant? Nein, sie stimmt genauso, und ist das Genaueste
was uns momentan zur Verfügung steht um die Klassische Physik zu vestehen. Es ist also mehr als
Genauso Relevant wie das, was momentan neues Passiert, die 'neulich' entdeckten Gravitationswellen
sind ein Beispiel dafür, Sie wurden akkurat erst durch besagte ART vorhergesagt, und nun eben gefunden.
Oder Satelliten, GPS würde nicht funktionieren mit Newton'scher Mechanik, erst über die ART wurde
es genau genug. Oder oder oder. Worauf ich hinaus möchte, nur weil es bereits bekannt ist, ist es
nicht langweilig ( \url{http://www.lesswrong.com/lw/j3/science\_as\_curiositystopper/} ). \\

\subsection*{Schluss}

Ich erzähle schon wieder viel zu viel, erwähne also noch kurz dass ich vorhabe die Welt zum
Positiven zu verändern und bin damit am Ende. Ich weiß nicht ob ich mich Bedanken soll dafür
dass Sie diesen langen Text gelesen haben, immerhin haben sie das Freiwillig getan, aber es
war durchaus ein wenig lang, möglicherweise ändere ich noch ein wenig die Formatierung damit
es einfacher wird zum lesen. Möglicherweise haben sie nun ein falsches Bild von mir, ich
habe irgendwie die Eigenschaft das relativ schnell ohne es zu bemerken hinzubekommen.
Lernen Sie mich doch einfach Kennen!



% - Gleichgesinnte zu treffen, kennenlernen, unterhalten \\
% - Interessante Gespräche führen und das gefühl haben, gemeinsam etwas zu schaffen \\
% - Persönliche Weiterbildung (und Hilfe für andere) \\
% - Um Nachhaltig die Welt zum Positiven zu verändern \\



\end{document}



