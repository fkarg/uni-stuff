\documentclass{scrartcl}
% \usepackage{german}
\usepackage[utf8]{inputenc}
\usepackage[german]{babel}
% \usepackage{amssymb} % what does it do?
% \usepackage{graphicx} % I can't do that yet
\usepackage{fancyhdr} % what does it do?
\usepackage{lastpage} % what does it do?
\setlength{\parskip}{\medskipamount} % thats reasonable
\setlength{\parindent}{0pt} % whatever that does
\usepackage{hyperref}
\usepackage[bottom=10em]{geometry}
\usepackage{lmodern}




%%%%%%%%%%%%%%%%%%%%%%%%
% Kopf- und Fusszeilen %
%%%%%%%%%%%%%%%%%%%%%%%%
\pagestyle{fancy}
\lhead{
    Felix Karg
}
\chead{Persönliche Motivation}
\rhead{
    \begin{tabular}{rr}
        \today{} \\
        Seite \thepage{} von \pageref{LastPage}
    \end{tabular}
}
\lfoot{}
\cfoot{}
\rfoot{}

%%%%%%%%%%%%%%%%%%%%%%%%
% Anfang des Dokuments %
%%%%%%%%%%%%%%%%%%%%%%%%
\begin{document}


\section*{Motivation für die Bewerbung}

Hallo an wer auch immer das Lesen wird. \\


\subsection*{Vorwort}

Da es weder Vorgaben für Inhalt, Umfang oder Format gibt, dachte ich mir, dass ich das Ganze
im Stil eines 'Stream of Consciousness' verfasse, also einfach niederschreibe, was mir gerade
durch den Kopf geht. Ich vermute, dass das eine Möglichkeit ist, einen relativ guten Einblick
in die Gedankenwelt einer anderen Person zu bekommen, auch wenn natürlich nicht alles in der
nötigen Geschwindigkeit aufgeschrieben werden kann, bzw. natürlich die Gedanken selten dermaßen
konkret formuliert sind.  \\


\subsection*{Informationen über mich}

Ich bin im Prinzip ein ganz normaler Informatik-Student an der Universität in Freiburg. Das
stimmt eventuell nicht so ganz, aber das möchte ich nicht vorwegnehmen. Ich interessiere mich
ehrlich gesagt für alles mögliche. Angefangen bei Physik und Mathematik über allgemeines
Planungsdenken, Philosophie, Chemie, Weltraum und Astronomie, natürlich Informatik, und viele
Teilbereiche von diesen.
Als ich spontan meine Freundin fragte, was für Dinge ich mögen würde, kamen folgende Antworten:
Programmieren, Informatik, KI, Mathematik, Physik, Strategie und Problemlösung,
Eigenständiges Denken und Lernen, Bücher, Raketen, Judo, Sudokus und Cookies, also ein weit
gefächertes Natur- bzw. Strukturwissenschaftliches Interesse.
Ich persönlich würde mich selbst als weltoffen beschreiben, reise im allgemeinen gerne
und diskutiere gerne, oder anders gesagt: Führe gerne interessante, anregende Gespräche über
alle möglichen Themen.


\subsection*{Erwartungen an ein Stipendium}
Was ich mir zum Ersten von einem Stipendium erhoffe ist mehr Leute mit ähnlichen Interessen zu
treffen, und dass sich mehr Möglichkeiten für interessanten Gespräche bieten. Allgemeiner
gesprochen noch: Ich erhoffe mir gewissermaßen Gleichgesinnte zu treffen, was meiner Erfahrung
nach nicht einfach ist. Andere weltoffene Menschen, oder solche, die einfach nur ihre Interessen
begeistert ausleben. Zudem glaube ich, dass gemeinsame Unternehmungen mit solchen Personen viel
Spaß machen würden, wie beispielsweise zusammen Vorträge zu besuchen oder sich bei gemeinsamen
Treffen auszutauschen. Ein zweiter großer Punkt ist meine persönliche Weiterbildung. Ich versuche
mich ständig weiterzubilden und zu verbessern, da es mich einfach Interessiert.
Auch halte ich die verbesserung meiner Fremdsprachenkenntnisse für sehr nützlich,
bei der internationalen Kommunikation wie auch in der Informatik ist Englisch die meiner Meinung
nach einfachste Art, sich mit interessanten Personen und Themen auf der ganzen Welt auseinanderzusetzen.
Der dritte Grund warum ich mich für ein Stipendium bewerbe ist, dass ich nicht nur Wissen
ansammeln, sondern es auch wieder weitergeben möchte. \\ Wenn möglich helfe ich natürlich meinen
Studienkollegen bei Fragen, und aufgrund meiner bereits etwas größeren Erfahrung in der Informatik
bin ich sogar einer der ersten Ansprechpartner bei größeren Problemen und helfe natürlich wo ich kann.

\subsection*{Interessen}
Wie bereits oben erwähnt interessiere ich mich für sehr verschiedene Gebiete, aus einem Grund:
Ich möchte gerne die Welt verstehen. Dafür sind nicht nur momentane Durchbrüche in der Forschung
von Belang, sondern auch großartigen Neuerungen der Vergangenheit. Ein Beispiel dafür wäre die
Allgemeine Relativitätstheorie. Obwohl ihre Entdeckung bereits über 100 Jahre zurück liegt ist
sie immer noch das Genaueste, was uns momentan zur Verfügung steht, um die klassische Physik
zu verstehen. Sie ist also durchaus genauso relevant wie Gebiete, an denen zurzeit geforscht
wird. Die „neulich“ entdeckten Gravitationswellen wären ohne die genaue Vorhersage eben genannter
ART heute wohl nicht bewiesen worden, genau wie Satelliten oder GPS möglich geworden sind.
Ich möchte also auch auf bewegende Forschungsdurchbrüche der Vergangenheit hinweisen. Für mich
persönlich ist die Wissenschaft und Forschung eine Methode, die Welt besser zu verstehen.
An dieser Stelle möchte ich auf einen interessanten Artikel zu dieser Sichtweise
verweisen, auf welchen ich gestoßen bin erst nachdem ich selbst dieser Meinung war:
\url{http://www.lesswrong.com/lw/j3/science\_as\_curiositystopper/}
).

\subsection*{Schluss}
Nach all diesen Informationen zu meiner Motivation, ein Stipendium zu beginnen, konnten sie
sich bestimmt ein konkretes Bild zu meiner Person bilden. Allerdings fallen mir immer noch
weitere Gründe ein, die ich oben nicht genannt habe. Am besten lernen Sie mich einfach selbst kennen!

\end{document}



