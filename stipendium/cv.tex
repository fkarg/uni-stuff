\documentclass{scrartcl}
\usepackage{german}
\usepackage[utf8]{inputenc}
\usepackage[german]{babel}
\usepackage{amssymb} % what does it do?
\usepackage{graphicx} % I can't do that yet
\usepackage{fancyhdr} % what does it do?
\usepackage{lastpage} % what does it do?
\setlength{\parskip}{\medskipamount} % thats reasonable
\setlength{\parindent}{0pt} % whatever that does


%%%%%%%%%%%%%%%%%%%%%%%%
% Kopf- und Fusszeilen %
%%%%%%%%%%%%%%%%%%%%%%%%
\pagestyle{fancy}
\lhead{
    Felix Karg
}
\chead{Detaillierter Lebenslauf}
\rhead{
    \begin{tabular}{rr}
        \today{} \\
        Seite \thepage{} von \pageref{LastPage}
    \end{tabular}
}
\lfoot{}
\cfoot{}
\rfoot{}

%%%%%%%%%%%%%%%%%%%%%%%%
% Anfang des Dokuments %
%%%%%%%%%%%%%%%%%%%%%%%%
\begin{document}


\section*{Persönliche Daten}

\begin{tabular}{|l|l|}
    \hline
    Vorname und Nachname & Felix Karg \\ \hline
    Staatszugehörigkeit & Deutschland \\ \hline
    E-Mail Adresse & f.karg10@gmail.com \\ \hline
    Telefonnummer & (+49) 176 8150 1822 \\ \hline
    Familienstand & ledig \\ \hline
    Geschwister & Drei, Ein direkt Jüngerer Bruder \\ \hline
    Geburtstag & 09.08.1998 \\ \hline
    Konfession & röm. kath. \\ \hline
    Adresse & Rheinstraße 23 \\
            & 79104 Freiburg \\ \hline
\end{tabular}

\section*{Schulausbildung}
\begin{tabular}{|l|l|}
    \hline
    \textbf{Seit 2016} & Universität Freiburg, Studium Informatik \\ \hline
    \textbf{2016-2014} & Fachoberschule (FOS) Friedberg \\ \hline
    \textbf{2014-2011} & Realschule Aichach \\ \hline
    \textbf{2011-2008} & Deutsch-Herren-Gymnasium Aichach \\ \hline
    \textbf{2008-2004} & Grundschule Aichach Nord \\ \hline
\end{tabular}

\section*{Weitere Tätigkeiten}
\begin{tabular}[c]{|l|l|}
    \hline
    \textbf{Zeitpunkt} & Veranstaltung / Event des Tages \\ \hline


    \textbf{2016 - Now } & Schwarze Pumpe, Schachverein in Freiburg \\ \hline
    \textbf{2010 - 2016} & BCA Abteilung Schach, Schachverein in Aichach \\ \hline
    \textbf{18.06.2016} & 3. Platz der Jugend Stadtmeisterschaft \\ \hline
    \textbf{11.07.2015} & 3. Platz der Jugend Stadtmeisterschaft \\ \hline

    \\ \hline

    \textbf{2014 - Now } & Judo \\ \hline
    \textbf{2007 - 2016} & Aichacher Ministranten \\ \hline
    \textbf{2012 - 2014} & Nachhilfe-Tutor für Mathe und Englisch \\ \hline
    \textbf{2013 - 2014} & Zeitungen Austragen \\ \hline
    \textbf{2011 - 2013} & Wahlfach Robotik \\ \hline
    \textbf{Schuljahr 2012 / 2013} & Erster Klassensprecher \\ \hline

\end{tabular}

\subsection*{Eventliste - nicht Vollständig}
\begin{tabular}{|l|l|}
    \hline

    \textbf{27.12. bis 30.12.2016} & Besuch des 33C3 \\ \hline
    \textbf{14.10. bis 16.10.2016} & Jugend Hackt Berlin - Awards wurden abgeschafft \\ \hline
    \textbf{20.08. bis 26.08.2016} & Mathecamp der Uni Augsburg \\ \hline
    \textbf{12.06. bis 14.06.2016} & Jugend Hackt Süd - AHA-Moment Award gewonnen \\ \hline
    \textbf{26.05. bis 29.05.2016} & Besuch der GPN \\ \hline
    \textbf{27.12. bis 30.12.2015} & Besuch des 32C3 \\ \hline
    \textbf{16.10. bis 18.10.2015} & Jugend Hackt Berlin - Best Code Award gewonnen \\ \hline
    \textbf{22.08. bis 28.08.2015} & Mathecamp der Uni Augsburg \\ \hline
    \textbf{13.08. bis 17.08.2015} & Besuch des Chaos Communication Camps bei Berlin \\ \hline
    \textbf{04.06. bis 07.06.2015} & Besuch der GPN \\ \hline
    \textbf{12.05. bis 14.05.2015} & Jugend Hackt Süd - Best Code Award gewonnen \\ \hline
    \textbf{Anfang 2015          } & Kennenlernen des OpenLabs Augsburg \\ \hline
    \textbf{21.05. bis 31.05.2013} & Praktikum als Fachinformatiker bei Leib-IT \\ \hline
    \textbf{25.03. bis 13.05.2013} & EDV-Praktikum bei der KJF Augsburg \\ \hline

\end{tabular}

\section*{Weitere Fähigkeiten}
\begin{tabular}{|l|l|}
    \hline
    \textbf{EDV- und} & Kenntnisse des Ersten Semesters Informatik \\
    \textbf{Computer-Kenntnisse} & Außerdem: Bereits weit Fortgeschrittene Programmier- \\
    & und Informatik-Kenntnisse durch verschiedene Events, \\
    & Projekte, sowie Selbststudium. \\ \hline
    \\ \hline
    \textbf{Sprachen} & Level \\ \hline
    Deutsch & Muttersprache \\ \hline
    Englisch & Kaum Unterschied zu Deutsch im Verstehen, minimal im Sprechen \\ \hline
    Japanisch & Es ist mir möglich mich vollkommen unkorrekt auszudrücken \\ \hline

\end{tabular}


\end{document}


