\documentclass{scrartcl}
\usepackage{german}
\usepackage[utf8]{inputenc}
\usepackage[german]{babel}
\usepackage{amssymb} % what does it do?
\usepackage{graphicx} % I can't do that yet
\usepackage{fancyhdr} % what does it do?
\usepackage{lastpage} % what does it do?
\setlength{\parskip}{\medskipamount} % thats reasonable
\setlength{\parindent}{0pt} % whatever that does
\usepackage{lmodern}


%%%%%%%%%%%%%%%%%%%%%%%%
% Kopf- und Fusszeilen %
%%%%%%%%%%%%%%%%%%%%%%%%
\pagestyle{fancy}
\lhead{
    Felix Karg
}
\chead{Detaillierter Lebenslauf}
\rhead{
    \begin{tabular}{rr}
        \today{} \\
        Seite \thepage{} von \pageref{LastPage}
    \end{tabular}
}
\lfoot{}
\cfoot{}
\rfoot{}

%%%%%%%%%%%%%%%%%%%%%%%%
% Anfang des Dokuments %
%%%%%%%%%%%%%%%%%%%%%%%%
\begin{document}


\section*{Persönlicher Detallierter Lebenslauf}

% \section*{Persönliche Daten}

% \vspace{3cm}
\bigskip
\bigskip
% \vfill

\begin{tabular}{|l|l|}
    % \hline
    \multicolumn{2}{l}{} \\
    \multicolumn{2}{l}{\LARGE \textbf{Persönliche Daten}} \\
    \multicolumn{2}{l}{} \\ \hline
    Vorname und Nachname & Felix Karg \\ \hline
    Staatszugehörigkeit & Deutschland \\ \hline
    E-Mail Adresse & f.karg10@gmail.com \\ \hline
    Telefonnummer & (+49) 176 8150 1822 \\ \hline
%    Familienstand & ledig \\ \hline
    Geschwister & Drei \\ \hline
    Geburtstag & 09.08.1998 \\ \hline
    Konfession & röm.-kath. \\ \hline
    Adresse & Rheinstraße 23 \\
            & 79104 Freiburg \\ \hline
% \end{tabular}
%
% \section*{Schulausbildung}
% \begin{tabular}{|l|l|}
%     \hline

    \multicolumn{2}{l}{} \\
    \multicolumn{2}{l}{\LARGE \textbf{Schulausbildung}} \\
    \multicolumn{2}{l}{} \\ \hline

    \textbf{Seit 2016} & Universität Freiburg, Studium Informatik \\ \hline
    \textbf{2016-2014} & Fachoberschule (FOS) Friedberg \\ \hline
    \textbf{2014-2011} & Realschule Aichach \\ \hline
    \textbf{2011-2008} & Deutsch-Herren-Gymnasium Aichach \\ \hline
    \textbf{2008-2004} & Grundschule Aichach Nord \\ \hline
% \end{tabular}
%
% \section*{Weitere Tätigkeiten}
% \begin{tabular}[c]{|l|l|}
%     \hline

    \multicolumn{2}{l}{} \\
    \multicolumn{2}{l}{\LARGE \textbf{Weitere Tätigkeiten}} \\
    \multicolumn{2}{l}{} \\ \hline

    \textbf{Zeitpunkt} & Veranstaltung / Event des Tages \\ \hline


    \textbf{2016 - Now } & Schwarze Pumpe, Schachverein in Freiburg \\ \hline
    \textbf{2010 - 2016} & BCA Abteilung Schach, Schachverein in Aichach \\ \hline
    \textbf{18.06.2016} & 3. Platz der Jugend Stadtmeisterschaft Schach \\ \hline
    \textbf{11.07.2015} & 3. Platz der Jugend Stadtmeisterschaft Schach \\ \hline

    \multicolumn{2}{|l|}{} \\ \hline

    \textbf{2014 - Now } & Judo \\ \hline
    \textbf{2007 - 2016} & Aichacher Ministranten \\ \hline
    \textbf{2012 - 2014} & Nachhilfe-Tutor für Mathe und Englisch \\ \hline
    \textbf{2013 - 2014} & Zeitungen Austragen \\ \hline
    \textbf{2011 - 2013} & Wahlfach Robotik \\ \hline
    % \textbf{Schuljahr 2012 / 2013} & Erster Klassensprecher \\ \hline

\end{tabular}


% \subsection*{Eventliste - nicht Vollständig}
% \begin{tabular}{|l|l|}
%     \hline
%     \textbf{27.12. bis 30.12.2016} & Besuch des 33C3 \\ \hline
%     \textbf{14.10. bis 16.10.2016} & Jugend Hackt Berlin - Awards wurden abgeschafft \\ \hline
%     \textbf{20.08. bis 26.08.2016} & Mathecamp der Uni Augsburg \\ \hline
%     \textbf{12.06. bis 14.06.2016} & Jugend Hackt Süd - AHA-Moment Award gewonnen \\ \hline
%     \textbf{26.05. bis 29.05.2016} & Besuch der GPN \\ \hline
%     \textbf{27.12. bis 30.12.2015} & Besuch des 32C3 \\ \hline
%     \textbf{16.10. bis 18.10.2015} & Jugend Hackt Berlin - Best Code Award gewonnen \\ \hline
%     \textbf{22.08. bis 28.08.2015} & Mathecamp der Uni Augsburg \\ \hline
%     \textbf{13.08. bis 17.08.2015} & Besuch des Chaos Communication Camps bei Berlin \\ \hline
%     \textbf{04.06. bis 07.06.2015} & Besuch der GPN \\ \hline
%     \textbf{12.05. bis 14.05.2015} & Jugend Hackt Süd - Best Code Award gewonnen \\ \hline
%     \textbf{Anfang 2015          } & Kennenlernen des OpenLabs Augsburg \\ \hline
%     \textbf{21.05. bis 31.05.2013} & Praktikum als Fachinformatiker bei Leib-IT \\ \hline
%     \textbf{25.03. bis 13.05.2013} & EDV-Praktikum bei der KJF Augsburg \\ \hline
% \end{tabular}


% \section*{Weitere Fähigkeiten}
% \begin{tabular}{|l|l|}
%     \hline
%     \textbf{EDV- und} & Kenntnisse des Ersten Semesters Informatik \\
%     \textbf{Computer-Kenntnisse} & Außerdem: Bereits weit Fortgeschrittene Programmier- \\
%     & und Informatik-Kenntnisse durch verschiedene Events, \\
%     & Projekte, sowie Selbststudium. \\ \hline
%     \\ \hline
%     \textbf{Sprachen} & Level \\ \hline
%     Deutsch & Muttersprache \\ \hline
%     Englisch & Kaum Unterschied zu Deutsch im Verstehen, minimal im Sprechen \\ \hline
%     Japanisch & Es ist mir möglich mich vollkommen unkorrekt auszudrücken \\ \hline
% \end{tabular}


\newpage


\section*{Ausformuliert}
Für Eine kurze Übersicht zu meiner Person siehe @{\em Einseitigen Lebenslauf}.


\subsection*{Von Anfang an ...}


% Angefangen in 86551 Aichach begonn meine Schullaufbahn 2004 kurz nachdem ich 6 Jahre alt wurde.
% Da das bereits sehr lange her und daher natürlich wenige subjekiv geprägte Erinnerungen sind,
% erlaube ich mir nichts dazu zu sagen. Anschließend wurde ich eingeschrieben auf dem DHG, dem
% örtlichen Gymnasium. Ich habe noch einige Erinnerungen an die Zeit damals, z.B. das Skilager
% oder verschiedene Lehrer, sowie den Büchereiclub, in dem ich erste Erfahrungen mit Programmieren
% gemacht habe. Nach der Siebten Klasse (Fachrichtung Technik) habe ich dann aufgrund teils Privater,
% teils Schulischer (schlechte Lateinnoten) auf die Realschule gewechselt. Im nachhinein muss ich
% sagen dass das vermutlich ein entscheidender Punkt in meinem Leben war, da ich dadurch viel
% Freizeit für das Programmieren und andere Interessen gewonnen habe, wie z.B. Schach oder
% Technik im allgemeinen. Ich habe mich in der Realschule Persönlich sowie Intellektuell mehrfach
% neu erfunden, was sich teilweise darin widergespiegelt hat dass ich mich teilweise lieber mit anderen
% Erwachsenen unterhalten habe als meinen Schulkameraden. Am ende meiner Realschulzeit hatte ich
% bereits nicht wenig Erfahrung in verschieden Programmiersprachen, und wusste Bescheid über die
% meisten grundlegenden Prinzipien. Das Interesse an fortgeschrittenen Themen war natürlich geweckt.
%
% Vermutlich haben die meisten Mitstudierenden (in sachen Programmieren) nun nach dem ersten Semester
% ein genauso gutes Verständnis dafür wie ich damals, wenn nicht sogar besser. Allerdings hatte ich
% natürlich damals weit mehr Zeit um mich damit auseinanderzusetzen, sowie ohne geregelten Plan
% bekam ich schnell auch fortgeschrittene Themen relativ am Anfang mit. Damals war ich der
% Meinung dass ich lieber direkt Anfangen würde zu Studieren anstatt auf die FOS zu gehen,
% allerdings waren die nächsten Zwei Jahre definitiv sehr Wichtig in meinem Entwicklungsprozess.
% Ich hatte ein halbes Jahr zuvor angefangen Judo zu Trainieren, einer der Gründe war
% Selbstfindung sowie Trainieren meiner Disziplin, und ja, ich mache immernoch regelmäßig Judo.
%
% Der Menschen als auch Schulwechsel tat mir insofern sehr gut, als dass ich bald angefangen
% habe mich für fortgeschrittene Gebiete der Informatik zu interessieren, oder mein bisher
% Größtes Projekt damals angefangen habe. Außerdem hatte ich vermutlich Ende der Zehnten
% Klasse den ersten Kontakt mit dem OpenLab Augsburg, der Lokalen vereinigung von Informatikern
% und Mathematikern, sowie anderen (Technik-) Interessierten oder Machern. Diese haben
% hin und wieder veranstaltungen Organisiert, und bald habe ich dort zum einen eifrig
% Mitgemacht oder war auch einfach so vor Ort um mich mit den Menschen dort zu unterhalten.
%
% Was mir auf alle fälle geholfen hat war die geniale Chaos-Community, bei der ich mich
% von Anfang an zugehörig fühlte (Fragen Sie mich nach der ersten begegnung). Nach dem
% ersten Jahr hatte ich auch angefangen die Theoretiker kennenzulernen, und habe mich
% gut mit den Mathematikern angefreundet. Ich habe dementsprechend beim Mathecamp
% teilgenommen, wurde total begeistert und war fortan (ich war Mathematik auch davor
% schon definitiv nicht abgeneigt gegenüber gewesen) begeistert von Uni-Mathematik
% oder allgemein Konzepten und ideen die eigentlich eher außer Reichweite schienen.
%
% Außerdem habe häufiger bei Jugend Hackt oder anderen (größeren) Chaos-Events
% teilgenommen, weswegen mein Vater ende der elften Klasse zu mir meinte dass ich anscheinend
% wirklich 'Aufgeblüht' sei dadurch, was vermutlich zurückblickend wirklich der Fall war.
% Auch durch die Zwölfte hindurch besuchte ich dann zweiwöchentlich die Universität Augsburg,
% Spezifisch die Mathematische Fakultät, um dem dort angebotenen 'Matheschülerzirkel'
% beizuwohnen sowie meine jeweils angesammelten Fragen über andere Themengebiete loszuwerden.
% Die sehr Theoretische seite (auch an der Informatik) war wirklicher Fortschritt für mich,
% und ich warte nur darauf dass es jeweils im Studium endlich drankommt, um es zu vertiefen.
%
% Auch durch die Zwölfte hindurch hatte ich einige Persönliche durchbrüche,
% da ich teils tiefe Einblicke in die verschiedensten Paradigmen zu verschiedensten Themen
% bekommen habe, so war es z.B. meine Ursprungszeit als Rationalist, und obwohl ich
% immernoch lange nicht angekommen bin, mache ich zumindest stetig Fortschritt. Genauso
% in meinen Kenntnissen der Mathematik, habe ich angefangen mich in Kategorientheorie
% einzulesen, einer Fundamental anderen Theorie gegenüber der üblichen Zermalo-Fränkel
% Mengentheorie (mit Auswahlaxiom). Auch meine Kenntnisse der Physik haben stetig
% zugenommen, und inzwischen meine ich zumindest sogar teile der Quantenmechanik oder
% Allgemeinen Relativitätstheorie zu verstehen, weswegen ich das irgendwann mal
% implementieren muss. Die Zwölfte war jedenfalls ein
% weitereres wichtiges Jahr für mich und ich weiß nicht wie wenige nur mich
% abgesehen vom Aussehen im verhältnis zur Zehnten wiedererkannt hätten.
%
% Inzwischen habe ich angefangen zu Studieren und muss nun nicht mehr Zwei Wochen
% warten bis ich irgendwelche Fragen stellen kann, sondern maximal bis zur nächsten
% Vorlesung. Auch ist es großartig andere Motivierte Gleichgesinnte zu treffen und
% gemeinsam Aufgaben zu bewältigen, für die man alleine um einiges Länger
% benötigen würde. Momentan habe ich außerdem eine Lese- und Lernphase, was
% im umkehrschluss heißt dass ich momentan wenig außerhalb der Aufgaben für
% die Universität Programmiere, was eigentlich schade ist.
%
% Aber das ist nur der Anfang. Was wird noch kommen?


Meine Schullaufbahn begann 2004 in der Grundschule Aichach, und vier Jahre später wechselte ich auf das DHG,
das örtliche Gymnasium. Ich erinnere mich daran, im schulinternen Büchereiclub meine ersten Programmiererfahrungen
gemacht zu haben.
Nach der siebten Klasse erfolgte aus schulischen und privaten Gründen mein Schulwechsel auf die Realschule.
Durch diese Entscheidung stand mir mehr Zeit zur Verfügung, welche ich mit Programmieren und anderen Interessen
wie Schach oder Robotik füllte. Während der nachfolgenden Zeit habe ich mich persönlich sowie intellektuell
mehrfach neu erfunden und meinen Horizont auf verschiedenen Gebieten erweitert. Am Ende meiner Realschulzeit
hatte ich bereits große Erfahrung in verschieden Programmiersprachen und wusste Bescheid über die meisten
grundlegenden Prinzipien.

Das Interesse an fortgeschritteneren Themen war natürlich geweckt.
Durch meine intensive Beschäftigung mit solchen Themen erreichte ich schon im Schulalter dieselben Grundkenntnisse,
wie sie im ersten Semester des Informatikstudiums gelehrt werden. Dadurch, dass ich mehr Zeit zur Verfügung hatte
als nur ein Halbjahr konnte ich diese sogar zusätzlich vertiefen und auch Ausblicke auf kompliziertere Themengebiete
erhaschen. Schon zu dieser Zeit war mein fester Plan ein Studium zu beginnen, dennoch waren auch die zwei folgenden
Jahre auf der FOS definitiv wichtig in meinen Entwicklungsprozess.
Seit 3 Jahren mache ich inzwischen Judo, diesen Sport habe ich zum Einen zur Selbstverwirklichung und zum Anderen zum
Trainieren meiner Selbstdisziplin begonnen.

Sowohl neue Kontakte als auch der Schulwechsel steigerten mein Interesse für fortgeschrittenere Gebiete der Informatik
und ich wagte mich an größere Projekte. Ende der zehnten Klasse besuchte ich zum ersten Mal das OpenLab Augsburg,
die lokalen Vereinigung von Informatikern, Mathematikern und anderen (Technik-)Interessierten. Beim Organisieren
von Veranstaltungen stand ich gern hilfsbereit zur Seite.
Weitere Events, bei denen ich begeistert teilnahm, waren Veranstaltungen der Chaos-Community, bei denen ich mich
von Anfang an zugehörig fühlte, wie beispielsweise Jugend Hackt, oder das Chaos-Camp. Mathematik übte schon immer
eine Faszination auf mich aus, und mit der Universitätsmathematik, die mir auf dem Mathecamp näher gebracht wurde,
begann ich auch Konzepte zu verstehen, die davor außer Reichweite schienen.

Mir nahestehende Personen wie mein Vater bemerkten in dieser Zeit, dass ich 'aufgeblüht' sei.
Auch durch die zwölfte hindurch besuchte ich dann zweiwöchentlich die Universität Augsburg, spezifisch die
Mathematische Fakultät, um dem dort angebotenen ’Matheschülerzirkel’ beizuwohnen sowie meine jeweils angesammelten
Fragen über andere Themengebiete loszuwerden. Auch durch die zwölfte hindurch hatte ich einige persönliche Durchbrüche,
da ich tiefe Einblicke in die verschiedensten Paradigmen bekommen habe. So war das zum Beispiel meine Ursprungszeit
als Rationalist, und ich mache stetig Fortschritt, genauso in meinen Kenntnissen der Mathematik.
Ich habe angefangen mich in Kategorientheorie einzulesen, einer fundamental anderen Theorie gegenüber der üblichen
Zermalo-Fränkel Mengentheorie (mit Auswahlaxiom). Auch meine Kenntnisse der Physik haben stetig zugenommen,
sogar auf dem Gebiet der Quantenmechanik oder der Allgemeinen Relativitätstheorie. In meinem laufenden Studium
nutze ich intensiv die Möglichkeit meinen Professoren auch unterrichtsfremde Fragen zu stellen und mich mit
anderen Gleichgesinnten über anstehende Aufgaben zu unterhalten, und freue mich darauf das in Zukunft weiter auszuführen.


\end{document}


