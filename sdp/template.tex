\documentclass[11pt,a4paper]{article}
\usepackage[utf8]{inputenc}
\usepackage{german}
\usepackage{xcolor}

\newcommand{\todo}[1]{\textcolor{red}{#1}}

\title{
  \Huge System Design Project \\
  \LARGE Zwischenbericht WS 2015/2016
}
\author{
  \Large Die Fantastischen Vier (\#70) \\
}

\begin{document}
\maketitle

\section{Gruppenmitglieder}
\begin{itemize}
  \item Victor Maier (4337015) (ESE)
  \item Paul Boe (4330649) (ESE)
  \item Christian Macha (Matrikelnummer) (MST)
  \item Felix Karg (4342014) (Info)
\end{itemize}

\section{Roboterkonzept}

Von Anfang an war der Plan, nicht den NXT-Baustein zu verwenden, sondern andere Lichtsensoren
und Arduinos zu verwenden, um damit eine bessere kompatibilität zum Arduino zu haben. Zum Meilenstein waren leider noch nicht ausreichend Teile da, weswegen wir kurzerhand
einen Lego-Roboter gebaut und eine Linienverfolgung Programmiert haben. Inzwischen sind die Teile
da, und wir sind am Überlegen, wie genau wir den Roboter letztendlich aufbauen. 
(Mehr dazu bei \emph{Weiteres Vorgehen})

\section{Softwarekonzept}

\emph{Optional:} Hier können Sie beschreiben, wie Sie die Programmierung des Roboters umgesetzt haben, z.B. durch angewandte Designpatterns.

\section{Fortschritt}

Bei der Präsentation des Robotersystems sollte generelle Funktionsfähigkeit insbesondere bei der
Linienverfolgung gezeigt werden. Problematisch war zunächt, dass der Roboter Schwierigkeiten beim
Hinauffahren von einer hohen Steigung hatte, da der Schwerpunkt zu weit nach hinten verlagert 
wurde. Weiterhin waren die Sensoren so angebracht, dass sie diesen Vorgang durch Bodenkontakt 
zusätzlich erschwerten. Keine Schwierigkeiten gab es bei rechtwinklingen Bahnabschnitten. 
Ausbaufähig wäre die benötigte Zeit um solche Abschnitte zu passieren, indem der Roboter durch 
Anpassung des Algorithmus schneller wendet und/oder danach beschleunigt. Zusammenfassend zeigte 
das System relativ geradlinige Bewegungen, was auf einen gut gewählten Abstand der Lichtsensoren
schließen lässt. Gleichzeitig sind Geschwindigkeitsanpassung noch möglich, wodurch nicht von einem 
idealen Abstand auszugehen ist. Von der Konstuktion der Räderanbrigung mussten die vorderen Räder 
stabiler angebracht werden, da sie sich zeitweise verkeilt haben, was jedes Mal korrigiert werden
musste. Sonst konnte die Zielsetzung für den Meilenstein gezeigt werden. Es gibt kein Foto davon,
weil der Linienverfolgende Roboter ziemlich direkt nach dem Meilenstein wieder Auseinandergebaut
wurde. Es gibt bereits Steckbretter mit mehreren Sensoren (Gyro, Kompass, diverse Lichtsensoren),
mit denen wir momentan versuchen, unseren eigentlich geplanten Roboter zu Programmieren.

\section{Fehleranalyse}
Die einzigen wirklichen Probleme die wir haben, ist dass wir vier uns eher selten treffen können,
um wirklich länger am Stück mal wieder am Roboter zu arbeiten, auch wenn wir uns an der Uni
teilweise Täglich sehen.

\section{Weiteres Vorgehen}

Eine weitere Idee unsererseits war, einen Roboter zu bauen, der Ähnlich einem Segway fährt,
weswegen wir noch einige Experimente mit dem Gyro vor uns haben. Den Kompass wollten wir verwenden,
um möglicherweise Teile der Strecke aufzuzeichnen um beim Zweiten mal durchfahren schneller zu sein.
Außerdem wollen wir so verhindern, uns ungeplant umzudrehen, und dies nicht zu bemerken.
Es ist abzusehen, wie weit uns diese Idee bringt.

\section{Arbeitsteilung}
Wir haben uns mehrfach getroffen, um über das Allgemeine Konzept, die Programmierung, neue Ideen,
etc. gemeinsam zu Diskutieren. Paul hatte mehr als wir anderen Ahnung von Arduino-Basierter 
Hardware, und hat uns dementsprechend beraten. Victor hat bereits einiges an 
Low-Level-Programmier-Erfahrung, und auch Felix hatte bereits mit Arduinos zu tun, wenn auch 
vermutlich nicht so viel wie Paul. Christian hatte bereits mehr Erfahrung mit solchen 
Teamprojekten, weswegen er Konstruktive Hinweise gab, oder auch die erste Version dieses 
Berichts schrieb.

\end{document}

